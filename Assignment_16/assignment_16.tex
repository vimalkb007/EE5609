\documentclass[journal,12pt]{IEEEtran}
\usepackage{longtable}
\usepackage{setspace}
\usepackage{gensymb}
\singlespacing
\usepackage[cmex10]{amsmath}
\newcommand\myemptypage{
	\null
	\thispagestyle{empty}
	\addtocounter{page}{-1}
	\newpage
}
\usepackage{amsthm}
\usepackage{mdframed}
\usepackage{mathrsfs}
\usepackage{txfonts}
\usepackage{stfloats}
\usepackage{bm}
\usepackage{cite}
\usepackage{cases}
\usepackage{subfig}

\usepackage{longtable}
\usepackage{multirow}

\usepackage{tikz}
\usetikzlibrary{automata, positioning, arrows}

\usepackage{enumitem}
\usepackage{mathtools}
\usepackage{steinmetz}
\usepackage{tikz}
\usepackage{circuitikz}
\usepackage{verbatim}
\usepackage{tfrupee}
\usepackage[breaklinks=true]{hyperref}
\usepackage{graphicx}
\usepackage{tkz-euclide}

\usetikzlibrary{calc,math}
\usepackage{listings}
\usepackage{color}                                            %%
\usepackage{array}                                            %%
\usepackage{longtable}                                        %%
\usepackage{calc}                                             %%
\usepackage{multirow}                                         %%
\usepackage{hhline}                                           %%
\usepackage{ifthen}                                           %%
\usepackage{lscape}     
\usepackage{multicol}
\usepackage{chngcntr}

\DeclareMathOperator*{\Res}{Res}

\renewcommand\thesection{\arabic{section}}
\renewcommand\thesubsection{\thesection.\arabic{subsection}}
\renewcommand\thesubsubsection{\thesubsection.\arabic{subsubsection}}

\renewcommand\thesectiondis{\arabic{section}}
\renewcommand\thesubsectiondis{\thesectiondis.\arabic{subsection}}
\renewcommand\thesubsubsectiondis{\thesubsectiondis.\arabic{subsubsection}}


\hyphenation{op-tical net-works semi-conduc-tor}
\def\inputGnumericTable{}                                 %%

\lstset{
	%language=C,
	frame=single, 
	breaklines=true,
	columns=fullflexible
}
\begin{document}
	\onecolumn
	
	\newtheorem{theorem}{Theorem}[section]
	\newtheorem{problem}{Problem}
	\newtheorem{proposition}{Proposition}[section]
	\newtheorem{lemma}{Lemma}[section]
	\newtheorem{corollary}[theorem]{Corollary}
	\newtheorem{example}{Example}[section]
	\newtheorem{definition}[problem]{Definition}
	
	\newcommand{\BEQA}{\begin{eqnarray}}
		\newcommand{\EEQA}{\end{eqnarray}}
	\newcommand{\define}{\stackrel{\triangle}{=}}
	\bibliographystyle{IEEEtran}
	\raggedbottom
	\setlength{\parindent}{0pt}
	\providecommand{\mbf}{\mathbf}
	\providecommand{\pr}[1]{\ensuremath{\Pr\left(#1\right)}}
	\providecommand{\qfunc}[1]{\ensuremath{Q\left(#1\right)}}
	\providecommand{\sbrak}[1]{\ensuremath{{}\left[#1\right]}}
	\providecommand{\lsbrak}[1]{\ensuremath{{}\left[#1\right.}}
	\providecommand{\rsbrak}[1]{\ensuremath{{}\left.#1\right]}}
	\providecommand{\brak}[1]{\ensuremath{\left(#1\right)}}
	\providecommand{\lbrak}[1]{\ensuremath{\left(#1\right.}}
	\providecommand{\rbrak}[1]{\ensuremath{\left.#1\right)}}
	\providecommand{\cbrak}[1]{\ensuremath{\left\{#1\right\}}}
	\providecommand{\lcbrak}[1]{\ensuremath{\left\{#1\right.}}
	\providecommand{\rcbrak}[1]{\ensuremath{\left.#1\right\}}}
	\theoremstyle{remark}
	\newtheorem{rem}{Remark}
	\newcommand{\sgn}{\mathop{\mathrm{sgn}}}
	\providecommand{\abs}[1]{\left\vert#1\right\vert}
	\providecommand{\res}[1]{\Res\displaylimits_{#1}} 
	\providecommand{\norm}[1]{\left\lVert#1\right\rVert}
	%\providecommand{\norm}[1]{\lVert#1\rVert}
	\providecommand{\mtx}[1]{\mathbf{#1}}
	\providecommand{\mean}[1]{E\left[ #1 \right]}
	\providecommand{\fourier}{\overset{\mathcal{F}}{ \rightleftharpoons}}
	%\providecommand{\hilbert}{\overset{\mathcal{H}}{ \rightleftharpoons}}
	\providecommand{\system}{\overset{\mathcal{H}}{ \longleftrightarrow}}
	%\newcommand{\solution}[2]{\textbf{Solution:}{#1}}
	\newcommand{\solution}{\noindent \textbf{Solution: }}
	\newcommand{\cosec}{\,\text{cosec}\,}
	\providecommand{\dec}[2]{\ensuremath{\overset{#1}{\underset{#2}{\gtrless}}}}
	\newcommand{\myvec}[1]{\ensuremath{\begin{pmatrix}#1\end{pmatrix}}}
	\newcommand{\mydet}[1]{\ensuremath{\begin{vmatrix}#1\end{vmatrix}}}
	\numberwithin{equation}{subsection}
	\makeatletter
	\@addtoreset{figure}{problem}
	\makeatother
	\let\StandardTheFigure\thefigure
	\let\vec\mathbf
	\renewcommand{\thefigure}{\theproblem}
	\def\putbox#1#2#3{\makebox[0in][l]{\makebox[#1][l]{}\raisebox{\baselineskip}[0in][0in]{\raisebox{#2}[0in][0in]{#3}}}}
	\def\rightbox#1{\makebox[0in][r]{#1}}
	\def\centbox#1{\makebox[0in]{#1}}
	\def\topbox#1{\raisebox{-\baselineskip}[0in][0in]{#1}}
	\def\midbox#1{\raisebox{-0.5\baselineskip}[0in][0in]{#1}}
	\vspace{3cm}
	\title{Assignment 16}
	\author{Vimal K B - AI20MTECH12001}
	\maketitle
	\bigskip
	\renewcommand{\thefigure}{\theenumi}
	\renewcommand{\thetable}{\theenumi}
	%
	Download the latex-tikz codes from 
	%
	\begin{lstlisting}
		https://github.com/vimalkb007/EE5609/tree/master/Assignment_16
	\end{lstlisting}
	\section{\textbf{Problem}}
	(UGC-june2015,77) : \\
	%
	Consider non-zero vector spaces $\vec{V_1}, \vec{V_2}, \vec{V_3}, \vec{V_4}$ and linear transformations $\phi_1 : \vec{V_1} \rightarrow \vec{V_2}$, $\phi_2 : \vec{V_2} \rightarrow \vec{V_3}$, $\phi_3 : \vec{V_3} \rightarrow \vec{V_4}$ such that $Ker(\phi) = \{0\}$, $Range(\phi_1) = Ker\{\phi_2\}$, $Range(\phi_2) = Ker\{\phi_3\}$, $Range(\phi_3) = \vec{V_4}$. Then \\
	
	\begin{enumerate}
		\item $\sum_{i=1}^{4} \ (-1)^{i} \ dim \ \vec{V_i} = 0$ \\
		\item $\sum_{i=2}^{4} \ (-1)^{i} \ dim \ \vec{V_i} > 0$ \\
		\item $\sum_{i=1}^{4} \ (-1)^{i} \ dim \ \vec{V_i} < 0$ \\
		\item $\sum_{i=1}^{4} \ (-1)^{i} \ dim \ \vec{V_i} \neq 0$
	\end{enumerate}
	
	
	\section{\textbf{Definition and Result used}}
	\begin{longtable}{|l|l|}
		\hline
		\multirow{3}{*}{Kernel and Nullity} 
		& \\
		& Given a linear transformation $L : \vec{V} \rightarrow \vec{W}$ between wo vector spaces $\vec{V}$ and \\ 
		& $\vec{W}$, the kernel of $L$ is the set of all vectors $\vec{v}$ of $\vec{V}$ for which $L(\vec{v}) = \vec{0}$, \\
		& where $\vec{0}$ denotes the zero vector in $\vec{W}$. i.e.\\
		& \\
		& \qquad \qquad \qquad $Ker(L) = \{\vec{v} \in \vec{V} \ |\ L(\vec{v}) = 0\}$ \\
		& \\
		& Nullity of the linear transformation is the dimension of the kernel of the linear \\
		& transformation i.e. \\
		& \\
		& \qquad \qquad \qquad $nullity(L) = dim(Ker(L))$ \\
		& \\
		\hline
		\multirow{3}{*}{Range and Rank} 
		& \\
		& Given a linear transformation $L : \vec{V} \rightarrow \vec{W}$ between wo vector spaces $\vec{V}$ and \\ 
		& $\vec{W}$, the range of $L$ is the set of all vectors $\vec{w}$ in $\vec{W}$ given as \\
		& \\
		& \qquad \qquad \qquad $Range(L) = \{\vec{w} \in \vec{W} \ |\ \vec{w} = L(\vec{v}), \vec{v} \in \vec{V}\}$ \\
		& \\
		& The rank of a linear transformation $L$ is the dimension of it's range, i.e. \\
		& \\
		& \qquad \qquad \qquad $rank(L) = dim(Range(L))$ \\
		& \\
		& \\
		\hline
		\multirow{3}{*}{Rank-Nullity Theorem} 
		& \\
		& Let $\vec{V}$, $\vec{W}$ be vector spaces, where $\vec{V}$ is finite dimensional. Let $L:\vec{V} \rightarrow \vec{W}$ be a \\
		& linear transformation. Then \\
		& \\
		&  \qquad \qquad  \qquad$rank(L) + nullity(L) = dim(\vec{V})$ \\
		& \\
		\hline
	\end{longtable}
	\section{\textbf{Solution}}
	\begin{longtable}{|l|l|}
		\hline
		\multirow{3}{*}{Inference from }   
		& \\ 
		& $Ker(\phi_1) = \{0\}$ \\the Given Data
		& \\
		& $\implies nullity(\phi_1) = 0$ \\
		& \\
		& \\
		& $Range(\phi_1) = Ker(\phi_2)$ \\
		& \\
		& $\implies rank(\phi_1) = nullity(\phi_2)$ \\
		& \\
		& \\
		& $Range(\phi_2) = Ker(\phi_3)$ \\
		& \\
		& $\implies rank(\phi_2) = nullity(\phi_3)$ \\
		& \\
		& \\
		& $Range(\phi_3) = \vec{V_4}$ \\
		& \\
		& $\implies rank(\phi_3) = dim(\vec{V_4})$ \\
		& \\
		& Now talking about the linear transformations we can use rank-nullity theorem to \\ & determine the corresponding dimensions of the vector space. \\
		& \\
		& \\
		& $\phi_1 : \vec{V_1} \rightarrow \vec{V_2}$ \\
		& \\
		& $\implies rank(\phi_1) + nullity(\phi_1) = dim(\vec{V_1})$ \\ 
		& $\implies rank(\phi_1) = dim(\vec{V_1})$ \qquad \qquad \qquad \qquad ($\because nullity(\phi_1) = 0$) \\
		& \\
		& \\
		& $\phi_2 : \vec{V_2} \rightarrow \vec{V_3}$ \\
		& \\
		& $\implies rank(\phi_2) + nullity(\phi_2) = dim(\vec{V_2})$ \\
		& $\implies rank(\phi_2) + rank(\phi_1) = dim(\vec{V_2})$ \qquad \qquad ($\because  rank(\phi_1) = nullity(\phi_2)$) \\
		& $\implies rank(\phi_2) + dim(\vec{V_1}) = dim(\vec{V_2})$ \qquad \qquad ($\because  rank(\phi_1) = dim(\vec{V_1})$) \\
		& \\
		& \\
		& $\phi_3 : \vec{V_3} \rightarrow \vec{V_4}$ \\
		& \\
		& $\implies rank(\phi_3) + nullity(\phi_3) = dim(\vec{V_3})$ \\
		& $\implies rank(\phi_3) + rank(\phi_2) = dim(\vec{V_3})$ \qquad \qquad ($\because  rank(\phi_2) = nullity(\phi_3)$) \\
		& $\implies rank(\phi_3) + dim(\vec{V_2}) - dim(\vec{V_1}) = dim(\vec{V_3})$     ($\because  rank(\phi_2) + dim(\vec{V_1}) = dim(\vec{V_2})$) \\
		& $\implies dim(\vec{V_4}) + dim(\vec{V_2}) - dim(\vec{V_1}) = dim(\vec{V_3})$ \qquad ($\because  rank(\phi_3) = dim(\vec{V_4})$) \\
		& \\
		& From the above equation we can infer that \\
		& \\
		& \qquad \qquad $dim(\vec{V_4}) + dim(\vec{V_2}) - dim(\vec{V_1}) - dim(\vec{V_3}) = 0$ \\
		& \\
		\hline
		\multirow{3}{*}{Option 1  } & \\
		& It is given that \\
		& \\
		& $\sum_{i=1}^{4} \ (-1)^{i} \ dim \ \vec{V_i} = 0$ \\
		& \\
		& $\implies - dim(\vec{V_1}) + dim(\vec{V_2}) - dim(\vec{V_3}) + dim(\vec{V_4}) = 0$ \\
		& \\
		& This statement we already proved above. \\
		& \\
		& $\therefore$ this statement is $\mathbf{True}$. \\
		&\\
		\hline
		\multirow{3}{*}{Option 2} & \\
		& It is given that \\
		& \\
		& $\sum_{i=2}^{4} \ (-1)^{i} \ dim \ \vec{V_i} > 0$ \\
		& \\
		& $\implies dim(\vec{V_2}) - dim(\vec{V_3}) + dim(\vec{V_4}) > 0$ \\
		& \\
		& Our original derived equation is \\
		& \\
		& \qquad \qquad $dim(\vec{V_4}) + dim(\vec{V_2}) - dim(\vec{V_1}) - dim(\vec{V_3}) = 0$ \\
		& \qquad $\implies$ $dim(\vec{V_2}) - dim(\vec{V_3}) + dim(\vec{V_4}) = dim(\vec{V_1})$ \\
		& \\
		& It is given in the question that the vector spaces are non-zero in nature. \\
		& \\
		& $\implies dim(\vec{V_1}) > 0$ \\
		& \\ 
		& \qquad \qquad $\therefore$ $dim(\vec{V_2}) - dim(\vec{V_3}) + dim(\vec{V_4}) > 0$\\
		& \\
		& $\therefore$ this statement is $\mathbf{True}$. \\
		&\\
		\hline
		\multirow{3}{*}{Option 3} & \\
		& It is given that \\
		& \\
		& $\sum_{i=1}^{4} \ (-1)^{i} \ dim \ \vec{V_i} < 0$ \\
		& \\
		& $\implies - dim(\vec{V_1}) + dim(\vec{V_2}) - dim(\vec{V_3}) +  dim(\vec{V_4}) < 0$ \\
		& \\
		& This is contrary to our original derived equation i.e. \\
		& \\
		& \qquad \qquad $dim(\vec{V_4}) + dim(\vec{V_2}) - dim(\vec{V_1}) - dim(\vec{V_3}) = 0$ \\
		& \\
		& $\therefore$ this statement is $\mathbf{False}$. \\
		&\\
		\hline
		\multirow{3}{*}{Option 4} & \\
		& It is given that \\
		& \\
		& $\sum_{i=1}^{4} \ (-1)^{i} \ dim \ \vec{V_i} \neq 0$ \\
		& \\
		& $\implies - dim(\vec{V_1}) + dim(\vec{V_2}) - dim(\vec{V_3}) +  dim(\vec{V_4}) \neq 0$ \\
		& \\
		& This is contrary to our original derived equation i.e. \\
		& \\
		& \qquad \qquad $dim(\vec{V_4}) + dim(\vec{V_2}) - dim(\vec{V_1}) - dim(\vec{V_3}) = 0$ \\
		& \\
		& $\therefore$ this statement is $\mathbf{False}$. \\
		&\\
		\hline
		\multirow{3}{*}{Conclusion} & \\
		& From our observation we see that \\
		&\\
		& Options 1) and 2) are True.\\
		& \\
		\hline
	\end{longtable}
	\section{\textbf{Example}}
	\begin{longtable}{|l|l|}
		\hline
		\multirow{3}{*}{Linear Tranform }   
		& \\ 
		& Let $L$ be a linear transformation $L: \mathbf{R}^3 \rightarrow \mathbf{R}^2$ be defined by \\
		& \\
		& \qquad \qquad \qquad $L\myvec{x_1\\x_2\\x_3} = \myvec{x_1 + x_2 \\ -2x_1 + x_2 - x_3}$ \\
		& \\
		\hline
		\multirow{3}{*}{Kernel and}   
		& \\ 
		& The above transformation can be written as \\ Nullity
		& \\
		& \qquad \qquad \qquad $L\myvec{x_1\\x_2\\x_3} = \myvec{1&1&0\\ -2&1&-1}\myvec{x_1\\x_2\\x_3}$ \\
		& \\
		& We will tkae the matrix $\myvec{1&1&0\\ -2&1&-1}$ do the row reduction as \\
		& \qquad \qquad \qquad $\myvec{1&1&0\\ -2&1&-1}$ \\
		& \\
		& \qquad $\xleftrightarrow[]{R_2 \leftarrow R_2 + 2R_1}$ $\myvec{1&1&0\\ 0&3&-1}$ \\
		& \\
		& \qquad $\xleftrightarrow[]{R_2 \leftarrow \frac{1}{3}R_2}$ $\myvec{1&1&0\\ 0&1&\frac{-1}{3}}$ \\
		& \\
		& \qquad $\xleftrightarrow[]{R_2 \leftarrow R_1 - R_2}$ $\myvec{1&0&\frac{1}{3}\\ 0&1&\frac{-1}{3}}$ \\
		& \\
		& We get \\
		& \qquad \qquad $x_1 + \frac{1}{3}x_3 = 0$ $\implies$ $x_1 = \frac{-1}{3}x_3$ \\
		& \\
		& \qquad \qquad $x_2 - \frac{1}{3}x_3 = 0$ $\implies$ $x_2 = \frac{1}{3}x_3$ \\
		& \\
		& $\therefore$ $Ker(L) = \left\{\myvec{\frac{-1}{3}\\\frac{1}{3}\\1}\right\}$ \\
		& \\
		& $\implies$ $nullity(L) = 1$ \\
		& \\
		\hline
		\multirow{3}{*}{Range and}   
		& \\ 
		& Range is defined as the span of columns. \\ Rank
		& For the Range, we take span of original pivot columns in our row reduced \\ 
		& echelon form. \\
		& \\
		& $\therefore$ $Range(L) = \left\{\myvec{1\\-2}, \myvec{1\\1}\right\} $ \\
		& \\
		& $\implies$ $rank(L) = 2$ \\
		& \\
		\hline
		\multirow{3}{*}{Rank-Nullity}   
		& \\ 
		& We know that $dim(\vec{R}^3) = 3$ \\ Theorem
		& \\
		& According to Rank-Nullity Theorem for the above defined transformation \\  
		& we should get\\ 
		& \\
		& \qquad \qquad \qquad $nullity(L) + rank(L) = dim(\vec{R}^3)$ \\
		& \\
		& And from the above values, we can see that the theorem is getting satisfied.\\
		& \\
		\hline
	\end{longtable}
		
\end{document}