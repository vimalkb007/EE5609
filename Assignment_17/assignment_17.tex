\documentclass[journal,12pt]{IEEEtran}
\usepackage{longtable}
\usepackage{setspace}
\usepackage{gensymb}
\singlespacing
\usepackage[cmex10]{amsmath}
\newcommand\myemptypage{
	\null
	\thispagestyle{empty}
	\addtocounter{page}{-1}
	\newpage
}
\usepackage{amsthm}
\usepackage{mdframed}
\usepackage{mathrsfs}
\usepackage{txfonts}
\usepackage{stfloats}
\usepackage{bm}
\usepackage{cite}
\usepackage{cases}
\usepackage{subfig}

\usepackage{longtable}
\usepackage{multirow}

\usepackage{tikz}
\usetikzlibrary{automata, positioning, arrows}

\usepackage{enumitem}
\usepackage{mathtools}
\usepackage{steinmetz}
\usepackage{tikz}
\usepackage{circuitikz}
\usepackage{verbatim}
\usepackage{tfrupee}
\usepackage[breaklinks=true]{hyperref}
\usepackage{graphicx}
\usepackage{tkz-euclide}

\usetikzlibrary{calc,math}
\usepackage{listings}
\usepackage{color}                                            %%
\usepackage{array}                                            %%
\usepackage{longtable}                                        %%
\usepackage{calc}                                             %%
\usepackage{multirow}                                         %%
\usepackage{hhline}                                           %%
\usepackage{ifthen}                                           %%
\usepackage{lscape}     
\usepackage{multicol}
\usepackage{chngcntr}

\DeclareMathOperator*{\Res}{Res}

\renewcommand\thesection{\arabic{section}}
\renewcommand\thesubsection{\thesection.\arabic{subsection}}
\renewcommand\thesubsubsection{\thesubsection.\arabic{subsubsection}}

\renewcommand\thesectiondis{\arabic{section}}
\renewcommand\thesubsectiondis{\thesectiondis.\arabic{subsection}}
\renewcommand\thesubsubsectiondis{\thesubsectiondis.\arabic{subsubsection}}


\hyphenation{op-tical net-works semi-conduc-tor}
\def\inputGnumericTable{}                                 %%

\lstset{
	%language=C,
	frame=single, 
	breaklines=true,
	columns=fullflexible
}
\begin{document}
	\onecolumn
	
	\newtheorem{theorem}{Theorem}[section]
	\newtheorem{problem}{Problem}
	\newtheorem{proposition}{Proposition}[section]
	\newtheorem{lemma}{Lemma}[section]
	\newtheorem{corollary}[theorem]{Corollary}
	\newtheorem{example}{Example}[section]
	\newtheorem{definition}[problem]{Definition}
	
	\newcommand{\BEQA}{\begin{eqnarray}}
		\newcommand{\EEQA}{\end{eqnarray}}
	\newcommand{\define}{\stackrel{\triangle}{=}}
	\bibliographystyle{IEEEtran}
	\raggedbottom
	\setlength{\parindent}{0pt}
	\providecommand{\mbf}{\mathbf}
	\providecommand{\pr}[1]{\ensuremath{\Pr\left(#1\right)}}
	\providecommand{\qfunc}[1]{\ensuremath{Q\left(#1\right)}}
	\providecommand{\sbrak}[1]{\ensuremath{{}\left[#1\right]}}
	\providecommand{\lsbrak}[1]{\ensuremath{{}\left[#1\right.}}
	\providecommand{\rsbrak}[1]{\ensuremath{{}\left.#1\right]}}
	\providecommand{\brak}[1]{\ensuremath{\left(#1\right)}}
	\providecommand{\lbrak}[1]{\ensuremath{\left(#1\right.}}
	\providecommand{\rbrak}[1]{\ensuremath{\left.#1\right)}}
	\providecommand{\cbrak}[1]{\ensuremath{\left\{#1\right\}}}
	\providecommand{\lcbrak}[1]{\ensuremath{\left\{#1\right.}}
	\providecommand{\rcbrak}[1]{\ensuremath{\left.#1\right\}}}
	\theoremstyle{remark}
	\newtheorem{rem}{Remark}
	\newcommand{\sgn}{\mathop{\mathrm{sgn}}}
	\providecommand{\abs}[1]{\left\vert#1\right\vert}
	\providecommand{\res}[1]{\Res\displaylimits_{#1}} 
	\providecommand{\norm}[1]{\left\lVert#1\right\rVert}
	%\providecommand{\norm}[1]{\lVert#1\rVert}
	\providecommand{\mtx}[1]{\mathbf{#1}}
	\providecommand{\mean}[1]{E\left[ #1 \right]}
	\providecommand{\fourier}{\overset{\mathcal{F}}{ \rightleftharpoons}}
	%\providecommand{\hilbert}{\overset{\mathcal{H}}{ \rightleftharpoons}}
	\providecommand{\system}{\overset{\mathcal{H}}{ \longleftrightarrow}}
	%\newcommand{\solution}[2]{\textbf{Solution:}{#1}}
	\newcommand{\solution}{\noindent \textbf{Solution: }}
	\newcommand{\cosec}{\,\text{cosec}\,}
	\providecommand{\dec}[2]{\ensuremath{\overset{#1}{\underset{#2}{\gtrless}}}}
	\newcommand{\myvec}[1]{\ensuremath{\begin{pmatrix}#1\end{pmatrix}}}
	\newcommand{\mydet}[1]{\ensuremath{\begin{vmatrix}#1\end{vmatrix}}}
	\numberwithin{equation}{subsection}
	\makeatletter
	\@addtoreset{figure}{problem}
	\makeatother
	\let\StandardTheFigure\thefigure
	\let\vec\mathbf
	\renewcommand{\thefigure}{\theproblem}
	\def\putbox#1#2#3{\makebox[0in][l]{\makebox[#1][l]{}\raisebox{\baselineskip}[0in][0in]{\raisebox{#2}[0in][0in]{#3}}}}
	\def\rightbox#1{\makebox[0in][r]{#1}}
	\def\centbox#1{\makebox[0in]{#1}}
	\def\topbox#1{\raisebox{-\baselineskip}[0in][0in]{#1}}
	\def\midbox#1{\raisebox{-0.5\baselineskip}[0in][0in]{#1}}
	\vspace{3cm}
	\title{Assignment 17}
	\author{Vimal K B - AI20MTECH12001}
	\maketitle
	\bigskip
	\renewcommand{\thefigure}{\theenumi}
	\renewcommand{\thetable}{\theenumi}
	%
	Download the latex-tikz codes from 
	%
	\begin{lstlisting}
		https://github.com/vimalkb007/EE5609/tree/master/Assignment_17
	\end{lstlisting}
	\section{\textbf{Problem}}
	(UGC-june2015,70) : \\
	%
	An $n\times n$ complex matrix $\vec{A}$ satisfies $\vec{A}^{k} = \vec{I}_n$. the $n\times n$ identity matrix, where $k$ is a positive integer $>$ 1. Suppose 1 is not an eigenvalue of $\vec{A}$. Then which of the following statements are necessarily true?\\
	
	\begin{enumerate}
		\item $\vec{A}$ is diagonalizable. \\
		\item $\vec{A}+\Vec{A}^2+...+\vec{A}^{k-1} = 0$, the $n\times n$ zero matrix. \\
		\item $tr(\vec{A})+tr(\Vec{A}^2)+...+tr(\vec{A}^{k-1}) = -n$ \\
		\item $\vec{A}^{-1}+\Vec{A}^{-2}+...+\vec{A}^{-(k-1)} = -\vec{I}_n$
	\end{enumerate}
	
	
	\section{\textbf{Definition and Result used}}
	\begin{longtable}{|l|l|}
		\hline
		\multirow{3}{*}{Minimal Polynomial} 
		& \\
		& The minimal polynomial $\mu_{\vec{A}}$ of an $n\times n$ matrix $\vec{A}$ over a field $\mathbf{F}$ is the \\
		& monic polynomial $P$ over the field $\mathbf{F}$ of least degree such that $P(\Vec{A}) = 0$. Any \\
		& other polynomial $Q$ with $Q(\vec{A}) = 0$ is polynomial multiple of $\mu_{\vec{A}}$. \\
		& \\
		\hline
		\multirow{3}{*}{Eigen Value and } 
		& \\
		& If $\lambda$ is an eigen value of matrix $\Vec{A}$ then $\lambda$ will also be the root of the minimal \\ Minimal Polynomial
		& polynomial $\mu_{\vec{A}}$.\\
		& \\
		\hline
		\multirow{3}{*}{Diagonalizability and} 
		& \\
		& If $\Vec{A}$ is an $n\times n$ matrix with $n$ distinct eigenvalues, then $\vec{A}$ is diagonalizable \\ Eigen Values
		& \\
		\hline
		\multirow{3}{*}{Polynomial and} 
		& \\
		& If a polynomial is of form $x^{k}-1$, it can be written as \\ it's Zeros
		& \\
		& \qquad \qquad \qquad $x^{k}-1$ = $(x - 1)(1 + x + x^2 + ... + x^{k-1})$\\
		& \\
		& The zeros to the given polynomial will be of the format \\
		& \\
		& \qquad \qquad \qquad $e^{\frac{n2\pi i}{k}}$ \qquad for $0 \leq n < k$. \\
		& \\
		& From this we can see that all the roots of the equation $x^{k}-1$ will be distinct. \\
		& \\
		\hline
	\end{longtable}
	\section{\textbf{Solution}}
	\begin{longtable}{|l|l|}
		\hline
		\multirow{3}{*}{Inference from }   
		& \\ 
		& We are given that \\the Given Data
		& \\
		& \qquad \qquad \qquad$\vec{A}^k = \vec{I}_n$ \\
		& \\
		& This can be written as \\
		& \\
		& \qquad \qquad \qquad$\vec{A}^k - \vec{I}_n = 0$ \\
		& \\
		& This resembles the polynomial equation of the form $x^{k}-1$, So we further write \\
		& the above equation as \\
		& \\
		& \qquad \qquad $\implies \vec{A}^k - \vec{I}_n = 0$ \\
		& \\
		& \qquad \qquad $\implies (\vec{A} - \vec{I}_n)(\vec{I}_n + \vec{A} + \vec{A}^2 + ... + \vec{A}^{k-1}) = 0$ \\
		& \\
		& Let $\mu_{\vec{A}}$ be the minimal polynomial of $\vec{A}$. \\
		& It is given that 1 is not an eigenvalue of $\vec{A}$. That means $\mu_{\vec{A}}$ cannot divide $(\vec{A} - \vec{I}_n)$.\\
		& \\
		& But $\mu_{\vec{A}}$ will be able to divide  $(\vec{I}_n + \vec{A} + \vec{A}^2 + ... + \vec{A}^{k-1})$ as it is a polynomial multiple of $\vec{A}$\\ 
		& \\
		& i.e. $(\vec{I}_n + \vec{A} + \vec{A}^2 + ... + \vec{A}^{k-1})$ is polynomial multiple of $\mu_{\vec{A}}$ \\
		& \\
		& \qquad \qquad  $\implies \vec{I}_n + \vec{A} + \vec{A}^2 + ... + \vec{A}^{k-1} = 0$ \\
		& \\
		& Since we know that $1 + x + x^2 + ... + x^{k-1}$ will have distinct roots which are not equal to 1. \\
		& \\
		\hline
		\multirow{3}{*}{Option 1  } & \\
		& We were able to find that $\implies \vec{I}_n + \vec{A} + \vec{A}^2 + ... + \vec{A}^{k-1}$ is a polynomial multiple of $\mu_{\vec{A}}$ \\
		& with $k-1$ distinct roots. Which implies that $\mu_{\vec{A}}$ will also have distinct roots. \\
		& \\
		& Since, there are distinct roots to the minimal polynomial, it implies that $\vec{A}$ will be \\ 
		& diagonalizable. \\
		& \\
		& $\therefore$ this statement is $\mathbf{True}$. \\
		&\\
		\hline
		\multirow{3}{*}{Option 2} & \\
		& We know that \\
		& \\
		& \qquad \qquad \qquad $\vec{I}_n + \vec{A} + \vec{A}^2 + ... + \vec{A}^{k-1} = 0$ \\
		& \\
		& \qquad \qquad $\implies \vec{A} + \vec{A}^2 + ... + \vec{A}^{k-1} = -\vec{I}_n$ \\
		& \\
		& $\therefore$ this statement is $\mathbf{False}$. \\
		&\\
		\hline
		\multirow{3}{*}{Option 3} & \\
		& We know that \\
		& \\
		& \qquad \qquad \qquad $\vec{I}_n + \vec{A} + \vec{A}^2 + ... + \vec{A}^{k-1} = 0$ \\
		& \\
		& \qquad \qquad $\implies \vec{A} + \vec{A}^2 + ... + \vec{A}^{k-1} = -\vec{I}_n$ \\
		& \\
		& Taking $trace()$ on both sides, we get \\
		& \\
		& \qquad \qquad $\implies tr(\vec{A} + \vec{A}^2 + ... + \vec{A}^{k-1}) = tr(-\vec{I}_n)$ \\
		& \\
		& \qquad \qquad $\implies tr(\vec{A}) + tr(\vec{A}^2) + ... + tr(\vec{A}^{k-1}) = tr(-\vec{I}_n)$ \qquad ($\because$ trace() is a linear function)\\
		& \\
		& \qquad \qquad $\implies tr(\vec{A}) + tr(\vec{A}^2) + ... + tr(\vec{A}^{k-1}) = -n$ \\
		& \\
		& $\therefore$ this statement is $\mathbf{True}$. \\
		&\\
		\hline
		\multirow{3}{*}{Option 4} & \\
		& We know that \\
		& \\
		& \qquad \qquad \qquad $\vec{I}_n + \vec{A} + \vec{A}^2 + ... + \vec{A}^{k-2} + \vec{A}^{k-1} = 0$ \\
		& \\
		& Multiply the whole equation with $\vec{A}^{-(k-1)}$. We get \\
		& \\
		& \qquad \qquad \qquad $\vec{A}^{-(k-1)} + \vec{A}^{1-(k-1)} + ... + \vec{A}^{k-2-(k-1)} + \vec{A}^{k-1-(k-1)} = 0$ \\
	    & \\
	    & \qquad \qquad $\implies \vec{A}^{-(k-1)} + \vec{A}^{1-(k-1)} + ... + \vec{A}^{-1} + \vec{I}_n = 0$ \\
	    & \\
	    & \qquad \qquad $\implies \vec{A}^{-1}+\Vec{A}^{-2}+...+\vec{A}^{-(k-1)} = -\vec{I}_n$ \\
	    & \\
		& $\therefore$ this statement is $\mathbf{True}$. \\
		&\\
		\hline
		\multirow{3}{*}{Conclusion} & \\
		& From our observation we see that \\
		&\\
		& Options 1), 3) and 4) are True.\\
		& \\
		\hline
	\end{longtable}
	\section{\textbf{Example}}
	\begin{longtable}{|l|l|}
		\hline
		\multirow{3}{*}{Complex Matrix }   
		& \\ 
		& Let the complex matrix $\vec{A}$ $=$ $\myvec{i&0 \\ 0&-i}$ \\ Example
		& \\
		& When $k = 4$, we get \\
		& \\
		& \qquad \qquad \qquad $\vec{A}^4 = \vec{I}_2$ \\
		& \\
		& \\
		& The eigen values of the matrix $\vec{A}$ are $-i$ and $+i$. \\
        & \\
        & Since, there are two distinct eigen values for the matrix $\vec{A}$,\\
        & $\vec{A}$ is diagonalizable. \\
        & \\
        & \\
        & Now checking the equation for $\vec{A}+\Vec{A}^2+...+\vec{A}^{k-1}$ \\
        & \\
        & \qquad \qquad \qquad $\vec{A}+\Vec{A}^2+\vec{A}^{3}$ \qquad ($\because$ here $k=4$) \\
        & \\
        & \qquad \qquad $\implies \myvec{i&0 \\ 0&-i} + \myvec{-1&0 \\ 0&-1} + \myvec{-i&0 \\ 0&i}$ \\
        & \\
        & \qquad \qquad $\implies \myvec{-1&0 \\ 0&-1} = -\vec{I}_2$  \\
        & \\
        & \\
        & Now checking the equation for $tr(\vec{A})+tr(\Vec{A}^2)+...+tr(\vec{A}^{k-1}) = -n$ \\
        & \\
        & \qquad \qquad \qquad $tr(\vec{A})+tr(\Vec{A}^2)+tr(\vec{A}^{3})$ \qquad ($\because$ here $k=4$) \\
        & \\
        & \qquad \qquad $\implies tr\myvec{i&0 \\ 0&-i} + tr\myvec{-1&0 \\ 0&-1} + tr\myvec{-i&0 \\ 0&i}$ \\
        & \\
        & \qquad \qquad $\implies 0+(-2)+0 = -2$\\
        & \\
		& \\
		& Now checking the equation for $\vec{A}^{-1}+\Vec{A}^{-2}+...+\vec{A}^{-(k-1)} = -\vec{I}_n$ \\
        & \\
        & \qquad \qquad \qquad $\vec{A}^{-1}+\Vec{A}^{-2}+\vec{A}^{-3}$ \qquad ($\because$ here $k=4$) \\
        & \\
        & \qquad \qquad $\implies \myvec{-i&0 \\ 0&i} + \myvec{-1&0 \\ 0&-1} + \myvec{i&0 \\ 0&-i}$ \\
        & \\
        & \qquad \qquad $\implies \myvec{-1&0 \\ 0&-1} = -\vec{I}_2$  \\
        & \\
		& \\
		\hline
	\end{longtable}
		
\end{document}