\documentclass[journal,12pt,twocolumn]{IEEEtran}

\usepackage{setspace}
\usepackage{gensymb}

\singlespacing


\usepackage[cmex10]{amsmath}

\usepackage{amsthm}

\usepackage{mathrsfs}
\usepackage{txfonts}
\usepackage{stfloats}
\usepackage{bm}
\usepackage{cite}
\usepackage{cases}
\usepackage{subfig}

\usepackage{longtable}
\usepackage{multirow}

\usepackage{enumitem}
\usepackage{mathtools}
\usepackage{steinmetz}
\usepackage{tikz}
\usepackage{circuitikz}
\usepackage{verbatim}
\usepackage{tfrupee}
\usepackage[breaklinks=true]{hyperref}

\usepackage{tkz-euclide}

\usetikzlibrary{calc,math}
\usepackage{listings}
\usepackage{color}                                            %%
\usepackage{array}                                            %%
\usepackage{longtable}                                        %%
\usepackage{calc}                                             %%
\usepackage{multirow}                                         %%
\usepackage{hhline}                                           %%
\usepackage{ifthen}                                           %%
\usepackage{lscape}     
\usepackage{multicol}
\usepackage{chngcntr}

\DeclareMathOperator*{\Res}{Res}

\renewcommand\thesection{\arabic{section}}
\renewcommand\thesubsection{\thesection.\arabic{subsection}}
\renewcommand\thesubsubsection{\thesubsection.\arabic{subsubsection}}

\renewcommand\thesectiondis{\arabic{section}}
\renewcommand\thesubsectiondis{\thesectiondis.\arabic{subsection}}
\renewcommand\thesubsubsectiondis{\thesubsectiondis.\arabic{subsubsection}}


\hyphenation{op-tical net-works semi-conduc-tor}
\def\inputGnumericTable{}                                 %%

\lstset{
	%language=C,
	frame=single, 
	breaklines=true,
	columns=fullflexible
}
\begin{document}
	
	
	\newtheorem{theorem}{Theorem}[section]
	\newtheorem{problem}{Problem}
	\newtheorem{proposition}{Proposition}[section]
	\newtheorem{lemma}{Lemma}[section]
	\newtheorem{corollary}[theorem]{Corollary}
	\newtheorem{example}{Example}[section]
	\newtheorem{definition}[problem]{Definition}
	
	\newcommand{\BEQA}{\begin{eqnarray}}
		\newcommand{\EEQA}{\end{eqnarray}}
	\newcommand{\define}{\stackrel{\triangle}{=}}
	\bibliographystyle{IEEEtran}
	\providecommand{\mbf}{\mathbf}
	\providecommand{\pr}[1]{\ensuremath{\Pr\left(#1\right)}}
	\providecommand{\qfunc}[1]{\ensuremath{Q\left(#1\right)}}
	\providecommand{\sbrak}[1]{\ensuremath{{}\left[#1\right]}}
	\providecommand{\lsbrak}[1]{\ensuremath{{}\left[#1\right.}}
	\providecommand{\rsbrak}[1]{\ensuremath{{}\left.#1\right]}}
	\providecommand{\brak}[1]{\ensuremath{\left(#1\right)}}
	\providecommand{\lbrak}[1]{\ensuremath{\left(#1\right.}}
	\providecommand{\rbrak}[1]{\ensuremath{\left.#1\right)}}
	\providecommand{\cbrak}[1]{\ensuremath{\left\{#1\right\}}}
	\providecommand{\lcbrak}[1]{\ensuremath{\left\{#1\right.}}
	\providecommand{\rcbrak}[1]{\ensuremath{\left.#1\right\}}}
	\theoremstyle{remark}
	\newtheorem{rem}{Remark}
	\newcommand{\sgn}{\mathop{\mathrm{sgn}}}
	\providecommand{\abs}[1]{\left\vert#1\right\vert}
	\providecommand{\res}[1]{\Res\displaylimits_{#1}} 
	\providecommand{\norm}[1]{\left\lVert#1\right\rVert}
	%\providecommand{\norm}[1]{\lVert#1\rVert}
	\providecommand{\mtx}[1]{\mathbf{#1}}
	\providecommand{\mean}[1]{E\left[ #1 \right]}
	\providecommand{\fourier}{\overset{\mathcal{F}}{ \rightleftharpoons}}
	%\providecommand{\hilbert}{\overset{\mathcal{H}}{ \rightleftharpoons}}
	\providecommand{\system}{\overset{\mathcal{H}}{ \longleftrightarrow}}
	%\newcommand{\solution}[2]{\textbf{Solution:}{#1}}
	\newcommand{\solution}{\noindent \textbf{Solution: }}
	\newcommand{\cosec}{\,\text{cosec}\,}
	\providecommand{\dec}[2]{\ensuremath{\overset{#1}{\underset{#2}{\gtrless}}}}
	\newcommand{\myvec}[1]{\ensuremath{\begin{pmatrix}#1\end{pmatrix}}}
	\newcommand{\mydet}[1]{\ensuremath{\begin{vmatrix}#1\end{vmatrix}}}
	\numberwithin{equation}{subsection}
	\makeatletter
	\@addtoreset{figure}{problem}
	\makeatother
	\let\StandardTheFigure\thefigure
	\let\vec\mathbf
	\renewcommand{\thefigure}{\theproblem}
	\def\putbox#1#2#3{\makebox[0in][l]{\makebox[#1][l]{}\raisebox{\baselineskip}[0in][0in]{\raisebox{#2}[0in][0in]{#3}}}}
	\def\rightbox#1{\makebox[0in][r]{#1}}
	\def\centbox#1{\makebox[0in]{#1}}
	\def\topbox#1{\raisebox{-\baselineskip}[0in][0in]{#1}}
	\def\midbox#1{\raisebox{-0.5\baselineskip}[0in][0in]{#1}}
	\vspace{3cm}
	\title{EE5609: Matrix Theory\\
		Assignment 10\\}
	\author{Vimal K B\\
		AI20MTECH12001}
	\maketitle
	\newpage
	\bigskip
	\renewcommand{\thefigure}{\theenumi}
	\renewcommand{\thetable}{\theenumi}
	\begin{abstract}
		This document explains the concept of linear operators, and significance of one-to-one and onto functions.
	\end{abstract}
	Download all solutions from 
	\begin{lstlisting}
		https://github.com/vimalkb007/EE5609/tree/master/Assignment_10
	\end{lstlisting}
	%
	%
	\section{Problem}
	Let $\mathbf{T}$ be a linear operator on the finite-dimensional space $\vec{V}$. Support there is a linear operator $\mathbf{U}$ on $\mathbf{V}$ such that $\mathbf{TU} = \vec{I}$. Prove that $\mathbf{T}$ is invertible and $\mathbf{U} = \mathbf{T}^{-1}$. Give an example which shows that this is false when $\mathbf{V}$ is not finite-dimensional.
	
	
	\section{Theorem}
	
	\begin{theorem}\label{thm1}
		Let $f$ be a function from $X$ into $Y$. We say that $f$ is invertible if there is a function $g$ from $Y$ to $X$ such that
		\begin{enumerate}
			\item $g \circ f$ is the identity function on $X$ i.e. $g \circ f$ = I. Here, g will be onto and f will be one-one.
			\item $f \circ g$ is the identity function on $Y$ i.e. $f \circ g$ = I. Here, f will be onto and g will be one-one.
		\end{enumerate}
	\end{theorem}
	
	\begin{theorem}\label{thm2}
		Let V and W be finite dimensional vector spaces such that  dim V = dim W. If T is a linear transformation from V into W, then the following are equivalent:
		\begin{enumerate}
			\item T is non-singular
			\item T is onto
		\end{enumerate}
		If any of the above two condition is satisfied then T is invertible.
	\end{theorem}
	
	
	
	\section{Solution}
	
	\begin{enumerate}
		\item We are given $\vec{V}$ which is a finite dimensional vector space, with the following linear operators defined as:-
		
		\begin{align}
			\mathbf{T} : \vec{V} \xrightarrow{} \vec{V} \\
			\mathbf{U} : \vec{V} \xrightarrow{} \vec{V}
		\end{align}
		
		The linear operators also satifies the condition
		
		\begin{align}
			\mathbf{T}\mathbf{U} = \vec{I} \label{eq1}
		\end{align}
		Where $\vec{I}$ is an Identity transformation. This identity transformation can be written as 
		\begin{align}
			\vec{I} : \vec{V} \xrightarrow{} \vec{V} \\
			\implies \mathbf{TU} : \vec{V} \xrightarrow{} \vec{V}\\
			\implies \mathbf{T}\left[ \mathbf{U}\left(\vec{V}\right)\right] = \vec{V}
		\end{align}
		From theorem $\eqref{thm1}$ we can say that $\mathbf{U}$ must be one-one and $\mathbf{V}$ must be onto.\\
		From theorem $\eqref{thm2}$ we can say that $\mathbf{T}$ is invertible.\\
		
		Now we know that
		\begin{align}
			\mathbf{T}\mathbf{T}^{-1} = \vec{I} \label{eq2}
		\end{align}
		
		Comparing $\eqref{eq1}$ and \eqref{eq2} we get
		\begin{align}
			\mathbf{T}\mathbf{T}^{-1} = \vec{I} = \mathbf{T}\mathbf{U}
		\end{align}
		Multiply both sides with $\mathbf{T}^{-1}$
		\begin{align}
			\mathbf{T}^{-1}\left(\mathbf{T}\mathbf{T}^{-1}\right) &= \mathbf{T}^{-1}\left(\mathbf{T}\mathbf{U}\right) \\
			\mathbf{T}^{-1}\vec{I} &= \left(\mathbf{T}^{-1}\mathbf{T}\right)\mathbf{U} \\
			\mathbf{T}^{-1} &= \vec{I}\mathbf{U}\\
			\therefore \mathbf{T}^{-1} &= \mathbf{U}
		\end{align}
		
		
		\item Let $\mathbf{D}$ be a differential operator $\mathbf{D} : \Vec{V} \xrightarrow{} \Vec{V}$, where $\vec{V}$ is a space of polynomial functions in one variable over $\mathbf{R}$.
		
		\begin{align}
			\mathbf{D}\left( c_0 + c_1x + ... + c_nx^{n}\right) = c_1 + c_2x + ... \nonumber\\+ c_nx^{n-1}
		\end{align}
		And $\mathbf{U} : \Vec{V} \xrightarrow{} \Vec{V}$ be linear operator such that
		\begin{align}
			\mathbf{U}\left( c_0 + c_1x + ... + c_nx^{n}\right) = c_0x + \frac{c_1x^{2}}{2} + ... \nonumber\\+ \frac{c_nx^{n+1}}{n+1}
		\end{align}
		Therefore, the linear operator $\mathbf{U}\mathbf{D} : \Vec{V} \xrightarrow{} \Vec{V}$ will be $\mathbf{U}\mathbf{D}\left( c_0 + c_1x + ... + c_nx^{n}\right)$
		\begin{align}\label{eq3}
			&= \mathbf{U}\left[\mathbf{D}\left( c_0 + c_1x + ... + c_nx^{n}\right)\right] \nonumber\\
			&= \mathbf{U}\left[c_1 + c_2x + ... + c_nx^{n-1}\right]\nonumber\\
			&= c_1x + \frac{c_2x^{2}}{2} + ... + \frac{c_nx^{n}}{n}\nonumber\\
			&= c_1x + c_2x^{2} + ... + c_nx^{n}\nonumber\\
			&\neq \vec{I}
		\end{align}
		Now, the linear operator $\mathbf{D}\mathbf{U} : \Vec{V} \xrightarrow{} \Vec{V}$ will be $\mathbf{D}\mathbf{U}\left( c_0 + c_1x + ... + c_nx^{n}\right)$
		\begin{align}\label{eq4}
			&= \mathbf{D}\left[\mathbf{U}\left( c_0 + c_1x + ... + c_nx^{n}\right)\right] \nonumber\\
			&= \mathbf{D}\left[c_0x + \frac{c_1x^{2}}{2} + ... + \frac{c_nx^{n+1}}{n+1}\right]\nonumber\\
			&= c_0 + \frac{2c_2x}{2} + ... + \frac{\left( n+1\right)c_nx^{n}}{n+1}\nonumber\\
			&= c_0 + c_1x + c_2x^{2} + ... + c_nx^{n}\nonumber\\
			&= \vec{I}
		\end{align}
		From $\eqref{eq3}$ and $\eqref{eq4}$ we see that $\mathbf{D}\mathbf{U} = \mathbf{I}$, but $\mathbf{U}\mathbf{D} \neq \mathbf{I}$. 
	\end{enumerate}
\end{document}