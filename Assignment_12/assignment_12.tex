\documentclass[journal,12pt,twocolumn]{IEEEtran}

\usepackage{setspace}
\usepackage{gensymb}

\singlespacing


\usepackage[cmex10]{amsmath}

\usepackage{amsthm}

\usepackage{mathrsfs}
\usepackage{txfonts}
\usepackage{stfloats}
\usepackage{bm}
\usepackage{cite}
\usepackage{cases}
\usepackage{subfig}

\usepackage{longtable}
\usepackage{multirow}

\usepackage{enumitem}
\usepackage{mathtools}
\usepackage{steinmetz}
\usepackage{tikz}
\usepackage{circuitikz}
\usepackage{verbatim}
\usepackage{tfrupee}
\usepackage[breaklinks=true]{hyperref}

\usepackage{tkz-euclide}

\usetikzlibrary{calc,math}
\usepackage{listings}
\usepackage{color}                                            %%
\usepackage{array}                                            %%
\usepackage{longtable}                                        %%
\usepackage{calc}                                             %%
\usepackage{multirow}                                         %%
\usepackage{hhline}                                           %%
\usepackage{ifthen}                                           %%
\usepackage{lscape}     
\usepackage{multicol}
\usepackage{chngcntr}

\DeclareMathOperator*{\Res}{Res}

\renewcommand\thesection{\arabic{section}}
\renewcommand\thesubsection{\thesection.\arabic{subsection}}
\renewcommand\thesubsubsection{\thesubsection.\arabic{subsubsection}}

\renewcommand\thesectiondis{\arabic{section}}
\renewcommand\thesubsectiondis{\thesectiondis.\arabic{subsection}}
\renewcommand\thesubsubsectiondis{\thesubsectiondis.\arabic{subsubsection}}


\hyphenation{op-tical net-works semi-conduc-tor}
\def\inputGnumericTable{}                                 %%

\lstset{
	%language=C,
	frame=single, 
	breaklines=true,
	columns=fullflexible
}
\begin{document}
	
	
	\newtheorem{theorem}{Theorem}[section]
	\newtheorem{problem}{Problem}
	\newtheorem{proposition}{Proposition}[section]
	\newtheorem{lemma}{Lemma}[section]
	\newtheorem{corollary}[theorem]{Corollary}
	\newtheorem{example}{Example}[section]
	\newtheorem{definition}[problem]{Definition}
	
	\newcommand{\BEQA}{\begin{eqnarray}}
		\newcommand{\EEQA}{\end{eqnarray}}
	\newcommand{\define}{\stackrel{\triangle}{=}}
	\bibliographystyle{IEEEtran}
	\providecommand{\mbf}{\mathbf}
	\providecommand{\pr}[1]{\ensuremath{\Pr\left(#1\right)}}
	\providecommand{\qfunc}[1]{\ensuremath{Q\left(#1\right)}}
	\providecommand{\sbrak}[1]{\ensuremath{{}\left[#1\right]}}
	\providecommand{\lsbrak}[1]{\ensuremath{{}\left[#1\right.}}
	\providecommand{\rsbrak}[1]{\ensuremath{{}\left.#1\right]}}
	\providecommand{\brak}[1]{\ensuremath{\left(#1\right)}}
	\providecommand{\lbrak}[1]{\ensuremath{\left(#1\right.}}
	\providecommand{\rbrak}[1]{\ensuremath{\left.#1\right)}}
	\providecommand{\cbrak}[1]{\ensuremath{\left\{#1\right\}}}
	\providecommand{\lcbrak}[1]{\ensuremath{\left\{#1\right.}}
	\providecommand{\rcbrak}[1]{\ensuremath{\left.#1\right\}}}
	\theoremstyle{remark}
	\newtheorem{rem}{Remark}
	\newcommand{\sgn}{\mathop{\mathrm{sgn}}}
	\providecommand{\abs}[1]{\left\vert#1\right\vert}
	\providecommand{\res}[1]{\Res\displaylimits_{#1}} 
	\providecommand{\norm}[1]{\left\lVert#1\right\rVert}
	%\providecommand{\norm}[1]{\lVert#1\rVert}
	\providecommand{\mtx}[1]{\mathbf{#1}}
	\providecommand{\mean}[1]{E\left[ #1 \right]}
	\providecommand{\fourier}{\overset{\mathcal{F}}{ \rightleftharpoons}}
	%\providecommand{\hilbert}{\overset{\mathcal{H}}{ \rightleftharpoons}}
	\providecommand{\system}{\overset{\mathcal{H}}{ \longleftrightarrow}}
	%\newcommand{\solution}[2]{\textbf{Solution:}{#1}}
	\newcommand{\solution}{\noindent \textbf{Solution: }}
	\newcommand{\cosec}{\,\text{cosec}\,}
	\providecommand{\dec}[2]{\ensuremath{\overset{#1}{\underset{#2}{\gtrless}}}}
	\newcommand{\myvec}[1]{\ensuremath{\begin{pmatrix}#1\end{pmatrix}}}
	\newcommand{\mydet}[1]{\ensuremath{\begin{vmatrix}#1\end{vmatrix}}}
	\numberwithin{equation}{subsection}
	\makeatletter
	\@addtoreset{figure}{problem}
	\makeatother
	\let\StandardTheFigure\thefigure
	\let\vec\mathbf
	\renewcommand{\thefigure}{\theproblem}
	\def\putbox#1#2#3{\makebox[0in][l]{\makebox[#1][l]{}\raisebox{\baselineskip}[0in][0in]{\raisebox{#2}[0in][0in]{#3}}}}
	\def\rightbox#1{\makebox[0in][r]{#1}}
	\def\centbox#1{\makebox[0in]{#1}}
	\def\topbox#1{\raisebox{-\baselineskip}[0in][0in]{#1}}
	\def\midbox#1{\raisebox{-0.5\baselineskip}[0in][0in]{#1}}
	\vspace{3cm}
	\title{EE5609: Matrix Theory\\
		Assignment 12\\}
	\author{Vimal K B\\
		AI20MTECH12001}
	\maketitle
	\newpage
	\bigskip
	\renewcommand{\thefigure}{\theenumi}
	\renewcommand{\thetable}{\theenumi}
	\begin{abstract}
		This document explains the relation between linear operators, and diagonalizability.
	\end{abstract}
	Download all solutions from 
	\begin{lstlisting}
	https://github.com/vimalkb007/EE5609/tree/master/Assignment_12
	\end{lstlisting}
	%
	%
	\section{Problem}
	Let $\mathbf{T}$ be the linear operator on $\mathbf{R}^{4}$ which is represented in the standard basis by the matrix \\
	\begin{align}
	    \myvec{0&0&0&0 \\ a&0&0&0 \\ 0&b&0&0 \\ 0&0&c&0} \nonumber
	\end{align}
	
	Under what conditions on $a$, $b$ and $c$ in $\mathbf{T}$ is diagonalizable?
	
	
	\section{Theorem}
	
	\begin{theorem}\label{thm1}
	    A linear operator $\mathbf{T}$ on a $n-$ dimensional space $\Vec{V}$ is diagonalizable, if and only if  $\mathbf{T}$ has an n distinct characteristic vectors (or) null spaces corresponding to the characteristic values. 
	\end{theorem}
	
	\begin{theorem}\label{thm2}
	    Let $\mathbf{T}$ be a linear operator on a finite-dimensional space $\Vec{V}$. Let $c_1, c_2,...,c_k$ be the distinct characteristic values of $\mathbf{T}$ and let $\mathbf{W_i}$ be the null space of $\left(\Vec{T}-c_{i}\Vec{I}\right)$. The following are equivalent:
	    \begin{enumerate}
	        \item $\mathbf{T}$ is diagonizable
	        \item $dim\ \Vec{W_1}+...+dim\ \Vec{W_k} = dim\ \Vec{V}$
	    \end{enumerate}
	\end{theorem}
	    
	\section{Solution}
	
	
	Let the given matrix be,
    
    \begin{align}
        \vec{A} = \myvec{0&0&0&0 \\ a&0&0&0 \\ 0&b&0&0 \\ 0&0&c&0}
    \end{align}
	
	As per theorem $\ref{thm1}$, we need to find the characteristic polynomial for the matrix $\Vec{A}$. Characteristic equation is given by $det\left ( x\Vec{I} - \Vec{A}\right)$.
	
	\begin{align}
	    det\left(x\Vec{I} - \Vec{A}\right) &= \begin{vmatrix}
                                                x-0 & 0 & 0 & 0 \\ 
                                                -a & x-0 & 0 & 0 \\
                                                0 & -b & x-0 & 0 \\
                                                0 & 0 & -c & x-0 \\
                                            \end{vmatrix}\\
        det\left(x\Vec{I} - \Vec{A}\right) &= x^{4}
	\end{align}
	
	The characteristic equation will be,
	
	\begin{align}
	    det\left(x\Vec{I} - \Vec{A}\right) &= 0\\
	    x^{4} &= 0 \label{eq1}
	\end{align}
	
	From $\eqref{eq1}$ we get the characteristic value as $c_1 = 0$ with a multiplicity of 4.\\ \\
	The basis for the characteristic value $c_1 = 0$ can be obtained by solving the equation
	\begin{align}
	    \left( \Vec{A} - c_1\Vec{I}\right) \Vec{x} &= \Vec{0}
	\end{align}
	i.e.
	\begin{align}
	    \left( \Vec{A} - (0)\Vec{I}\right) \Vec{x} &= \Vec{0} \\
	    \myvec{0&0&0&0 \\ a&0&0&0 \\ 0&b&0&0 \\ 0&0&c&0}\myvec{x\\y\\z\\t} &= \Vec{0} 
	\end{align}
	
	Solving the above equation we get
	
	\begin{align}\label{eq2}
	    ax = 0,\ by = 0,\ cz = 0 \\ 
	\end{align}
	
	From theorem $\ref{thm2}$, for $\mathbf{T}$ to be diagonalizable, the null space $\Vec{W_1}$ of $\vec{A}$ has $dim\ \Vec{W_1} = 4$, which is only possible if in equation $\eqref{eq2}$ we get $a = 0$, $b = 0$ and $c = 0$.\\
	$\therefore$ $\Vec{A}$ is diagonalizable only if
	\begin{align}
	    a \ = \ b  \ = \ c \ = \ 0
	\end{align}
	i.e.\ $\Vec{A}$ is a zero matrix.
\end{document}