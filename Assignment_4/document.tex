\documentclass[journal,12pt,twocolumn]{IEEEtran}

\usepackage{setspace}
\usepackage{gensymb}

\singlespacing


\usepackage[cmex10]{amsmath}

\usepackage{amsthm}

\usepackage{mathrsfs}
\usepackage{txfonts}
\usepackage{stfloats}
\usepackage{bm}
\usepackage{cite}
\usepackage{cases}
\usepackage{subfig}

\usepackage{longtable}
\usepackage{multirow}

\usepackage{enumitem}
\usepackage{mathtools}
\usepackage{steinmetz}
\usepackage{tikz}
\usepackage{circuitikz}
\usepackage{verbatim}
\usepackage{tfrupee}
\usepackage[breaklinks=true]{hyperref}
\usepackage{graphicx}
\usepackage{tkz-euclide}

\usetikzlibrary{calc,math}
\usepackage{listings}
\usepackage{color}                                            %%
\usepackage{array}                                            %%
\usepackage{longtable}                                        %%
\usepackage{calc}                                             %%
\usepackage{multirow}                                         %%
\usepackage{hhline}                                           %%
\usepackage{ifthen}                                           %%
\usepackage{lscape}     
\usepackage{multicol}
\usepackage{chngcntr}

\DeclareMathOperator*{\Res}{Res}

\renewcommand\thesection{\arabic{section}}
\renewcommand\thesubsection{\thesection.\arabic{subsection}}
\renewcommand\thesubsubsection{\thesubsection.\arabic{subsubsection}}

\renewcommand\thesectiondis{\arabic{section}}
\renewcommand\thesubsectiondis{\thesectiondis.\arabic{subsection}}
\renewcommand\thesubsubsectiondis{\thesubsectiondis.\arabic{subsubsection}}


\hyphenation{op-tical net-works semi-conduc-tor}
\def\inputGnumericTable{}                                 %%

\lstset{
	%language=C,
	frame=single, 
	breaklines=true,
	columns=fullflexible
}
\begin{document}
	
	
	\newtheorem{theorem}{Theorem}[section]
	\newtheorem{problem}{Problem}
	\newtheorem{proposition}{Proposition}[section]
	\newtheorem{lemma}{Lemma}[section]
	\newtheorem{corollary}[theorem]{Corollary}
	\newtheorem{example}{Example}[section]
	\newtheorem{definition}[problem]{Definition}
	
	\newcommand{\BEQA}{\begin{eqnarray}}
		\newcommand{\EEQA}{\end{eqnarray}}
	\newcommand{\define}{\stackrel{\triangle}{=}}
	\bibliographystyle{IEEEtran}
	\providecommand{\mbf}{\mathbf}
	\providecommand{\pr}[1]{\ensuremath{\Pr\left(#1\right)}}
	\providecommand{\qfunc}[1]{\ensuremath{Q\left(#1\right)}}
	\providecommand{\sbrak}[1]{\ensuremath{{}\left[#1\right]}}
	\providecommand{\lsbrak}[1]{\ensuremath{{}\left[#1\right.}}
	\providecommand{\rsbrak}[1]{\ensuremath{{}\left.#1\right]}}
	\providecommand{\brak}[1]{\ensuremath{\left(#1\right)}}
	\providecommand{\lbrak}[1]{\ensuremath{\left(#1\right.}}
	\providecommand{\rbrak}[1]{\ensuremath{\left.#1\right)}}
	\providecommand{\cbrak}[1]{\ensuremath{\left\{#1\right\}}}
	\providecommand{\lcbrak}[1]{\ensuremath{\left\{#1\right.}}
	\providecommand{\rcbrak}[1]{\ensuremath{\left.#1\right\}}}
	\theoremstyle{remark}
	\newtheorem{rem}{Remark}
	\newcommand{\sgn}{\mathop{\mathrm{sgn}}}
	\providecommand{\abs}[1]{\left\vert#1\right\vert}
	\providecommand{\res}[1]{\Res\displaylimits_{#1}} 
	\providecommand{\norm}[1]{\left\lVert#1\right\rVert}
	%\providecommand{\norm}[1]{\lVert#1\rVert}
	\providecommand{\mtx}[1]{\mathbf{#1}}
	\providecommand{\mean}[1]{E\left[ #1 \right]}
	\providecommand{\fourier}{\overset{\mathcal{F}}{ \rightleftharpoons}}
	%\providecommand{\hilbert}{\overset{\mathcal{H}}{ \rightleftharpoons}}
	\providecommand{\system}{\overset{\mathcal{H}}{ \longleftrightarrow}}
	%\newcommand{\solution}[2]{\textbf{Solution:}{#1}}
	\newcommand{\solution}{\noindent \textbf{Solution: }}
	\newcommand{\cosec}{\,\text{cosec}\,}
	\providecommand{\dec}[2]{\ensuremath{\overset{#1}{\underset{#2}{\gtrless}}}}
	\newcommand{\myvec}[1]{\ensuremath{\begin{pmatrix}#1\end{pmatrix}}}
	\newcommand{\mydet}[1]{\ensuremath{\begin{vmatrix}#1\end{vmatrix}}}
	\numberwithin{equation}{subsection}
	\makeatletter
	\@addtoreset{figure}{problem}
	\makeatother
	\let\StandardTheFigure\thefigure
	\let\vec\mathbf
	\renewcommand{\thefigure}{\theproblem}
	\def\putbox#1#2#3{\makebox[0in][l]{\makebox[#1][l]{}\raisebox{\baselineskip}[0in][0in]{\raisebox{#2}[0in][0in]{#3}}}}
	\def\rightbox#1{\makebox[0in][r]{#1}}
	\def\centbox#1{\makebox[0in]{#1}}
	\def\topbox#1{\raisebox{-\baselineskip}[0in][0in]{#1}}
	\def\midbox#1{\raisebox{-0.5\baselineskip}[0in][0in]{#1}}
	\vspace{3cm}



\title{EE5609 Assignment 3}
\author{Vimal K B \\Roll No - AI20MTECH14002}

\maketitle
\newpage
%\tableofcontents
\bigskip

\renewcommand{\thefigure}{\theenumi}
\renewcommand{\thetable}{\theenumi}

\begin{abstract}
	This assignment involves finding the determinant of the given matrix.
\end{abstract}

The python solution code for this problem can be downloaded from

\begin{lstlisting}
	https://github.com/vimalkb007/EE5609/blob/master/Assignment_4/codes/assignment4_solution.py
\end{lstlisting}

\section{\textbf{Problem Statement}}
Evaluate 
$\begin{vmatrix}
	\cos\alpha \cos\beta &\cos\alpha \sin\beta &-\sin\alpha \\ 
	-\sin\beta & \cos\beta &0 \\ \sin\alpha\cos\beta&\sin\alpha\sin\beta&\cos\alpha
\end{vmatrix}.$\\
		
	
\section{\textbf{Solution}}

We first multiply either the rows or the columns, and then try taking the common element out.

\begin{align}
	\mydet{
		\cos\alpha \cos\beta & \cos\alpha \sin\beta & -\sin\alpha \\ 
		-\sin\beta & \cos\beta & 0 \\ 
		\sin\alpha\cos\beta & \sin\alpha\sin\beta & \cos\alpha
	} \nonumber
\xleftrightarrow[C_3 \leftarrow (\sin\alpha) C_3]{C_3 \leftarrow (\cos\alpha) C_3}
\end{align}

\begin{align}\label{eq1}
	\left(\frac{1}{\sin\alpha\cos\alpha}\right)
	\mydet{
		\cos\alpha \cos\beta &\cos\alpha \sin\beta & -\sin^{2}\alpha \cos\alpha\\ 
		-\sin\beta & \cos\beta & 0 \\ 
		\sin\alpha\cos\beta & \sin\alpha\sin\beta & \cos^{2}\alpha\sin\alpha
	}	
\end{align}

From \eqref{eq1} $R_1$ we take out common element $\cos\alpha$. 
And from row $R_2$ we take out common element $\sin\alpha$

\begin{align}\label{eq2}
	\mydet{
		\cos\beta & \sin\beta & -\sin^{2}\alpha\\ 
		-\sin\beta & \cos\beta & 0 \\ 
		\cos\beta & \sin\beta & \cos^{2}\alpha
	}\nonumber \xleftrightarrow[]{R_1 \leftarrow R_1 - R_3}	\\
	\mydet{
		0 & 0 & -\sin^{2}\alpha - \cos^{2}\alpha\\ 
		-\sin\beta & \cos\beta & 0 \\ 
		\cos\beta & \sin\beta & \cos^{2}\alpha
	}
\end{align}

Now, we can expand the determinant from row 1 in \eqref{eq2}, and we get

\begin{align}
	\left(-\sin^{2}\alpha - \cos^{2}\alpha\right) \left(-\sin^{2}\beta - \cos^{2}\beta\right)\nonumber \\
	\implies \left(\sin^{2}\alpha + \cos^{2}\alpha\right) \left(\sin^{2}\beta + \cos^{2}\beta\right) = 1
\end{align}

    Therefore, the determinant of the matrix is 1.
\end{document}