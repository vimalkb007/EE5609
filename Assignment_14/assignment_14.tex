\documentclass[journal,12pt,twocolumn]{IEEEtran}

\usepackage{setspace}
\usepackage{gensymb}

\singlespacing


\usepackage[cmex10]{amsmath}

\usepackage{amsthm}

\usepackage{mathrsfs}
\usepackage{txfonts}
\usepackage{stfloats}
\usepackage{bm}
\usepackage{cite}
\usepackage{cases}
\usepackage{subfig}

\usepackage{longtable}
\usepackage{multirow}

\usepackage{enumitem}
\usepackage{mathtools}
\usepackage{steinmetz}
\usepackage{tikz}
\usepackage{circuitikz}
\usepackage{verbatim}
\usepackage{tfrupee}
\usepackage[breaklinks=true]{hyperref}

\usepackage{tkz-euclide}

\usetikzlibrary{calc,math}
\usepackage{listings}
\usepackage{color}                                            %%
\usepackage{array}                                            %%
\usepackage{longtable}                                        %%
\usepackage{calc}                                             %%
\usepackage{multirow}                                         %%
\usepackage{hhline}                                           %%
\usepackage{ifthen}                                           %%
\usepackage{lscape}     
\usepackage{multicol}
\usepackage{chngcntr}

\DeclareMathOperator*{\Res}{Res}

\renewcommand\thesection{\arabic{section}}
\renewcommand\thesubsection{\thesection.\arabic{subsection}}
\renewcommand\thesubsubsection{\thesubsection.\arabic{subsubsection}}

\renewcommand\thesectiondis{\arabic{section}}
\renewcommand\thesubsectiondis{\thesectiondis.\arabic{subsection}}
\renewcommand\thesubsubsectiondis{\thesubsectiondis.\arabic{subsubsection}}


\hyphenation{op-tical net-works semi-conduc-tor}
\def\inputGnumericTable{}                                 %%

\lstset{
	%language=C,
	frame=single, 
	breaklines=true,
	columns=fullflexible
}
\begin{document}
	
	
	\newtheorem{theorem}{Theorem}[section]
	\newtheorem{problem}{Problem}
	\newtheorem{proposition}{Proposition}[section]
	\newtheorem{lemma}{Lemma}[section]
	\newtheorem{corollary}[theorem]{Corollary}
	\newtheorem{example}{Example}[section]
	\newtheorem{definition}[problem]{Definition}
	
	\newcommand{\BEQA}{\begin{eqnarray}}
		\newcommand{\EEQA}{\end{eqnarray}}
	\newcommand{\define}{\stackrel{\triangle}{=}}
	\bibliographystyle{IEEEtran}
	\providecommand{\mbf}{\mathbf}
	\providecommand{\pr}[1]{\ensuremath{\Pr\left(#1\right)}}
	\providecommand{\qfunc}[1]{\ensuremath{Q\left(#1\right)}}
	\providecommand{\sbrak}[1]{\ensuremath{{}\left[#1\right]}}
	\providecommand{\lsbrak}[1]{\ensuremath{{}\left[#1\right.}}
	\providecommand{\rsbrak}[1]{\ensuremath{{}\left.#1\right]}}
	\providecommand{\brak}[1]{\ensuremath{\left(#1\right)}}
	\providecommand{\lbrak}[1]{\ensuremath{\left(#1\right.}}
	\providecommand{\rbrak}[1]{\ensuremath{\left.#1\right)}}
	\providecommand{\cbrak}[1]{\ensuremath{\left\{#1\right\}}}
	\providecommand{\lcbrak}[1]{\ensuremath{\left\{#1\right.}}
	\providecommand{\rcbrak}[1]{\ensuremath{\left.#1\right\}}}
	\theoremstyle{remark}
	\newtheorem{rem}{Remark}
	\newcommand{\sgn}{\mathop{\mathrm{sgn}}}
	\providecommand{\abs}[1]{\left\vert#1\right\vert}
	\providecommand{\res}[1]{\Res\displaylimits_{#1}} 
	\providecommand{\norm}[1]{\left\lVert#1\right\rVert}
	%\providecommand{\norm}[1]{\lVert#1\rVert}
	\providecommand{\mtx}[1]{\mathbf{#1}}
	\providecommand{\mean}[1]{E\left[ #1 \right]}
	\providecommand{\fourier}{\overset{\mathcal{F}}{ \rightleftharpoons}}
	%\providecommand{\hilbert}{\overset{\mathcal{H}}{ \rightleftharpoons}}
	\providecommand{\system}{\overset{\mathcal{H}}{ \longleftrightarrow}}
	\newcommand\R{\mathbb{R}}
	%\newcommand{\solution}[2]{\textbf{Solution:}{#1}}
	\newcommand{\solution}{\noindent \textbf{Solution: }}
	\newcommand{\cosec}{\,\text{cosec}\,}
	\providecommand{\dec}[2]{\ensuremath{\overset{#1}{\underset{#2}{\gtrless}}}}
	\newcommand{\myvec}[1]{\ensuremath{\begin{pmatrix}#1\end{pmatrix}}}
	\newcommand{\mydet}[1]{\ensuremath{\begin{vmatrix}#1\end{vmatrix}}}
	
	\numberwithin{equation}{subsection}
	\makeatletter
	\@addtoreset{figure}{problem}
	\makeatother
	\let\StandardTheFigure\thefigure
	\let\vec\mathbf
	\renewcommand{\thefigure}{\theproblem}
	\def\putbox#1#2#3{\makebox[0in][l]{\makebox[#1][l]{}\raisebox{\baselineskip}[0in][0in]{\raisebox{#2}[0in][0in]{#3}}}}
	\def\rightbox#1{\makebox[0in][r]{#1}}
	\def\centbox#1{\makebox[0in]{#1}}
	\def\topbox#1{\raisebox{-\baselineskip}[0in][0in]{#1}}
	\def\midbox#1{\raisebox{-0.5\baselineskip}[0in][0in]{#1}}
	\vspace{3cm}
	\title{EE5609: Matrix Theory\\
		Assignment 14\\}
	\author{Vimal K B\\
		AI20MTECH12001}
	\maketitle
	\newpage
	\bigskip
	\renewcommand{\thefigure}{\theenumi}
	\renewcommand{\thetable}{\theenumi}
	\begin{abstract}
		This document proves the property of projection.
	\end{abstract}
	Download all solutions from 
	\begin{lstlisting}
		https://github.com/vimalkb007/EE5609/tree/master/Assignment_14
	\end{lstlisting}
	%
	%
	\section{\textbf{Problem}}
	Prove that if $\mathbf{E}$ is the projection on $\mathbf{R}$ along $\mathbf{N}$, then $(\mathbf{I-E})$ is the projection on $\mathbf{N}$ along $\mathbf{R}$ 
	
	
	\section{\textbf{Theorem}}
	
	\begin{theorem}\label{thm1}
		If $\vec{V}$ is a vector space, a projection of $\vec{V}$ is a linear operator $\mathbf{E}$ on $\vec{V}$ such that $\mathbf{E}^2$ = $\mathbf{E}$. Let $\mathbf{R}$ be the range and let $\mathbf{N}$ be the nullspace of $\mathbf{E}$. Then the vector space $\vec{V}$ can be written as $\Vec{V} = \mathbf{R} \bigoplus \mathbf{N}$. This operator is called as projection on $\mathbf{R}$ along $\mathbf{N}$. 
	\end{theorem}
	
	\section{\textbf{Solution}}
	
	It is given that $\mathbf{E}$ is the projection. From thorem $\ref{thm1}$, the linear operator $\mathbf{E}$ will satisfy $\mathbf{E}^{2} = \mathbf{E}$. Let's check whether $\mathbf{I-E}$ is also a projection. 
	
	\begin{align}\label{eq1}
		(\Vec{I}-\vec{E})^{2} &= \Vec{I}^{2} + \Vec{E}^{2} - 2\Vec{I}\Vec{E} \nonumber \\
		&= \Vec{I} + \Vec{E} - 2\Vec{E} \nonumber \\
		&= (\Vec{I} - \Vec{E})
	\end{align}
	
	From $\eqref{eq1}$, we can say that $(\Vec{I} - \Vec{E})$ is also a projector. But $(\Vec{I} - \Vec{E})$ is called as the "Complementary Projector", i.e.
	\begin{align}
		range(\Vec{I} - \Vec{E}) &= null(\Vec{E}) \label{eq2}\\
		null(\Vec{I} - \Vec{E}) &= range(\Vec{E}) \label{eq3}
	\end{align}
	
	Lets take a vector $\Vec{v}$ such that $\Vec{Ev} = 0$, where $\vec{v}$ is in the null space of $\vec{E}$. Then, 
	
	\begin{align}
		(\Vec{I} - \Vec{E})\vec{v} &= \vec{v} - \vec{v}\vec{E} \nonumber \\
		&= \vec{v}
	\end{align}
	In other words, any $\Vec{v}$ in the nullspace of $\Vec{E}$ is also in the range of $(\Vec{I} - \Vec{E})$.  
	We know that any $\vec{x} \in range(\Vec{I} - \Vec{E})$ is characterized by
	\begin{align}
		\Vec{x} &= (\Vec{I} - \Vec{E})\Vec{v} \quad \textit{, for some $\Vec{v}$} \nonumber \\
		&= \Vec{v} - \Vec{Ev} \nonumber \\
		&= - \ (\Vec{Ev} - \Vec{v})
	\end{align}
	Now we need to check if $\Vec{x}$ is in the nullspace of $\Vec{E}$. i.e. $\Vec{Ex} = 0$
	\begin{align}
		\Vec{E}(- \ (\Vec{Ev} - \Vec{v})) &= -(\Vec{E}^{2}\Vec{v} - \Vec{E}\Vec{v}) \nonumber \\
		&= -(\Vec{E}\Vec{v} - \Vec{E}\Vec{v}) \quad (\textit{$\because$ $\vec{E}$ is a projection}) \nonumber \\
		&= 0
	\end{align}
	
	Thus, if $\vec{x} \in range(\Vec{I} - \Vec{E})$, then $\Vec{x} \in null(\Vec{E})$.\\
	
	Therefore, we can say that $null(\Vec{E}) = range(\Vec{I} - \Vec{E})$. \\
	
	We can use the same argument as above for proving $\eqref{eq3}$, by taking $\Vec{E} = \Vec{I} - (\Vec{I} - \Vec{E})$.\\
	
	$\therefore$ we can say that $(\Vec{I} - \Vec{E})$ is the projection on $\mathbf{N}$ along $\mathbf{R}$.
	
\end{document}