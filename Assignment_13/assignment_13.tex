\documentclass[journal,12pt,twocolumn]{IEEEtran}

\usepackage{setspace}
\usepackage{gensymb}

\singlespacing


\usepackage[cmex10]{amsmath}

\usepackage{amsthm}

\usepackage{mathrsfs}
\usepackage{txfonts}
\usepackage{stfloats}
\usepackage{bm}
\usepackage{cite}
\usepackage{cases}
\usepackage{subfig}

\usepackage{longtable}
\usepackage{multirow}

\usepackage{enumitem}
\usepackage{mathtools}
\usepackage{steinmetz}
\usepackage{tikz}
\usepackage{circuitikz}
\usepackage{verbatim}
\usepackage{tfrupee}
\usepackage[breaklinks=true]{hyperref}

\usepackage{tkz-euclide}

\usetikzlibrary{calc,math}
\usepackage{listings}
\usepackage{color}                                            %%
\usepackage{array}                                            %%
\usepackage{longtable}                                        %%
\usepackage{calc}                                             %%
\usepackage{multirow}                                         %%
\usepackage{hhline}                                           %%
\usepackage{ifthen}                                           %%
\usepackage{lscape}     
\usepackage{multicol}
\usepackage{chngcntr}

\DeclareMathOperator*{\Res}{Res}

\renewcommand\thesection{\arabic{section}}
\renewcommand\thesubsection{\thesection.\arabic{subsection}}
\renewcommand\thesubsubsection{\thesubsection.\arabic{subsubsection}}

\renewcommand\thesectiondis{\arabic{section}}
\renewcommand\thesubsectiondis{\thesectiondis.\arabic{subsection}}
\renewcommand\thesubsubsectiondis{\thesubsectiondis.\arabic{subsubsection}}


\hyphenation{op-tical net-works semi-conduc-tor}
\def\inputGnumericTable{}                                 %%

\lstset{
	%language=C,
	frame=single, 
	breaklines=true,
	columns=fullflexible
}
\begin{document}
	
	
	\newtheorem{theorem}{Theorem}[section]
	\newtheorem{problem}{Problem}
	\newtheorem{proposition}{Proposition}[section]
	\newtheorem{lemma}{Lemma}[section]
	\newtheorem{corollary}[theorem]{Corollary}
	\newtheorem{example}{Example}[section]
	\newtheorem{definition}[problem]{Definition}
	
	\newcommand{\BEQA}{\begin{eqnarray}}
		\newcommand{\EEQA}{\end{eqnarray}}
	\newcommand{\define}{\stackrel{\triangle}{=}}
	\bibliographystyle{IEEEtran}
	\providecommand{\mbf}{\mathbf}
	\providecommand{\pr}[1]{\ensuremath{\Pr\left(#1\right)}}
	\providecommand{\qfunc}[1]{\ensuremath{Q\left(#1\right)}}
	\providecommand{\sbrak}[1]{\ensuremath{{}\left[#1\right]}}
	\providecommand{\lsbrak}[1]{\ensuremath{{}\left[#1\right.}}
	\providecommand{\rsbrak}[1]{\ensuremath{{}\left.#1\right]}}
	\providecommand{\brak}[1]{\ensuremath{\left(#1\right)}}
	\providecommand{\lbrak}[1]{\ensuremath{\left(#1\right.}}
	\providecommand{\rbrak}[1]{\ensuremath{\left.#1\right)}}
	\providecommand{\cbrak}[1]{\ensuremath{\left\{#1\right\}}}
	\providecommand{\lcbrak}[1]{\ensuremath{\left\{#1\right.}}
	\providecommand{\rcbrak}[1]{\ensuremath{\left.#1\right\}}}
	\theoremstyle{remark}
	\newtheorem{rem}{Remark}
	\newcommand{\sgn}{\mathop{\mathrm{sgn}}}
	\providecommand{\abs}[1]{\left\vert#1\right\vert}
	\providecommand{\res}[1]{\Res\displaylimits_{#1}} 
	\providecommand{\norm}[1]{\left\lVert#1\right\rVert}
	%\providecommand{\norm}[1]{\lVert#1\rVert}
	\providecommand{\mtx}[1]{\mathbf{#1}}
	\providecommand{\mean}[1]{E\left[ #1 \right]}
	\providecommand{\fourier}{\overset{\mathcal{F}}{ \rightleftharpoons}}
	%\providecommand{\hilbert}{\overset{\mathcal{H}}{ \rightleftharpoons}}
	\providecommand{\system}{\overset{\mathcal{H}}{ \longleftrightarrow}}
	\newcommand\R{\mathbb{R}}
	%\newcommand{\solution}[2]{\textbf{Solution:}{#1}}
	\newcommand{\solution}{\noindent \textbf{Solution: }}
	\newcommand{\cosec}{\,\text{cosec}\,}
	\providecommand{\dec}[2]{\ensuremath{\overset{#1}{\underset{#2}{\gtrless}}}}
	\newcommand{\myvec}[1]{\ensuremath{\begin{pmatrix}#1\end{pmatrix}}}
	\newcommand{\mydet}[1]{\ensuremath{\begin{vmatrix}#1\end{vmatrix}}}
	\numberwithin{equation}{subsection}
	\makeatletter
	\@addtoreset{figure}{problem}
	\makeatother
	\let\StandardTheFigure\thefigure
	\let\vec\mathbf
	\renewcommand{\thefigure}{\theproblem}
	\def\putbox#1#2#3{\makebox[0in][l]{\makebox[#1][l]{}\raisebox{\baselineskip}[0in][0in]{\raisebox{#2}[0in][0in]{#3}}}}
	\def\rightbox#1{\makebox[0in][r]{#1}}
	\def\centbox#1{\makebox[0in]{#1}}
	\def\topbox#1{\raisebox{-\baselineskip}[0in][0in]{#1}}
	\def\midbox#1{\raisebox{-0.5\baselineskip}[0in][0in]{#1}}
	\vspace{3cm}
	\title{EE5609: Matrix Theory\\
		Assignment 13\\}
	\author{Vimal K B\\
		AI20MTECH12001}
	\maketitle
	\newpage
	\bigskip
	\renewcommand{\thefigure}{\theenumi}
	\renewcommand{\thetable}{\theenumi}
	\begin{abstract}
		This document checks for the characteristic and minimal polynomial.
	\end{abstract}
	Download all solutions from 
	\begin{lstlisting}
	https://github.com/vimalkb007/EE5609/tree/master/Assignment_13
	\end{lstlisting}
	%
	%
	\section{Problem}
	Let $\vec{A}$ and $\vec{B}$ be $n\times n$ matrices over the field $\mathbf{F}$. If the matrices $\vec{AB}$ and $\vec{BA}$ have the same characteristic values. Do they have the same characteristic polynomial? Do they have the same minimal polynomial? 
 	
	
	\section{Theorem}
	
	\begin{theorem}\label{thm1}
	    If $\mathbf{T}$ is a linear operator on a finite-dimensional space $\vec{V}$ and c is a characteristic value of $\mathbf{T}$, then the operator $(\mathbf{T} - c\vec{I})$ is singular, i.e.
	    \begin{align}
	    	det\left(\mathbf{T}-c\vec{I}\right) = 0 \nonumber
	    \end{align} 
	\end{theorem}
	    
	\section{Solution}
	
	To prove that $\vec{AB}$ and $\vec{BA}$ have the same characteristic values in $\mathbf{F}$, we can use theorem $\ref{thm1}$ and show
	\begin{align}\label{eq1}
		det\left(\mathbf{AB}-c\vec{I}\right) = det\left(\mathbf{BA}-c\vec{I}\right) = 0
	\end{align}
	
	Let us consider,
	\begin{align}
		det(\vec{AB}-c\vec{I}) &= 0\nonumber\\
		&= det(\vec{I})\ det(\vec{AB}-c\vec{I})\nonumber\\
		&= det(\vec{A^{-1}A})\ det(\vec{AB}-c\vec{I})\nonumber\\
		&= det(\vec{A}^{-1})\ det(\vec{AB}-c\vec{I})\ det(\vec{A})\nonumber\\
		&= det(\vec{A}^{-1}(\vec{AB}-c\vec{I})\vec{A})\nonumber\\
		&= det((\vec{A}^{-1}\vec{AB}-\vec{A}^{-1}(c\vec{I}))\vec{A})\nonumber\\
		&= det((\vec{A}^{-1}\vec{AB}-c\vec{A}^{-1}\vec{I})\vec{A})\nonumber\\
		&= det((\vec{IB}-c\vec{A}^{-1})\vec{A})\nonumber\\
		&= det((\vec{B}-c\vec{A}^{-1})\vec{A})\nonumber\\
		&= det(\vec{BA}-c\vec{A}^{-1}\vec{A})\nonumber\\
		&= det(\vec{BA}-c\vec{I})	
	\end{align}

	And we already know that, 
	\begin{align}
		det(\vec{BA}-c\vec{I}) &= 0
	\end{align}

	$\therefore$ $\vec{AB}$ and $\vec{BA}$ have the same characteristic values. \\
	
	After proving $\eqref{eq1}$, we can say that the two polynomials of degree $n$ with exactly the same roots, will be equal in nature. So, the characteristic polynomials are equal. \\
	
	Even though the characteristic polynomials are same, they may not necessarily be the same minimal polynomial.
	
	Lets take the below examples
	
	\begin{align}
		\vec{A} = \myvec{0&1\\0&0} \\
		\vec{B} = \myvec{0&0\\0&1}
	\end{align}

	We get the matrices $\vec{AB}$ and $\vec{BA}$ as
	
	\begin{align}
		\vec{AB} = \myvec{0&1\\0&0} \\
		\vec{BA} = \myvec{0&0\\0&0}
	\end{align}

	Then $\vec{AB} = \vec{A}$, whereas $\vec{BA}$ is the zero matrix. Since $\vec{A^2} = 0$ and $\vec{A} \neq 0$, the minimal polynomial of $\vec{AB}$ is $x^2$, whereas the minimal polynomial of $\vec{BA}$ is $x$.
	
	$\therefore$, $\vec{AB}$ and $\vec{BA}$ have the same characteristic polynomial, but the minimal polynomials are not the same.
	
	
\end{document}