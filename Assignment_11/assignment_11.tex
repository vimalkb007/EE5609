\documentclass[journal,12pt,twocolumn]{IEEEtran}
%
\usepackage{setspace}
\usepackage{gensymb}
\usepackage{siunitx}
\usepackage{tkz-euclide} 
\usepackage{textcomp}
\usepackage{standalone}
\usetikzlibrary{calc}

%\doublespacing
\singlespacing

%\usepackage{graphicx}
%\usepackage{amssymb}
%\usepackage{relsize}
\usepackage[cmex10]{amsmath}
%\usepackage{amsthm}
%\interdisplaylinepenalty=2500
%\savesymbol{iint}
%\usepackage{txfonts}
%\restoresymbol{TXF}{iint}
%\usepackage{wasysym}
\usepackage{amsthm}
%\usepackage{iithtlc}
\usepackage{mathrsfs}
\usepackage{txfonts}
\usepackage{stfloats}
\usepackage{bm}
\usepackage{cite}
\usepackage{cases}
\usepackage{subfig}
%\usepackage{xtab}
\usepackage{longtable}
\usepackage{multirow}
%\usepackage{algorithm}
%\usepackage{algpseudocode}
\usepackage{enumitem}
\usepackage{mathtools}
\usepackage{steinmetz}
\usepackage{tikz}
\usepackage{circuitikz}
\usepackage{verbatim}
\usepackage{tfrupee}
\usepackage[breaklinks=true]{hyperref}
%\usepackage{stmaryrd}
\usepackage{tkz-euclide} % loads  TikZ and tkz-base
%\usetkzobj{all}
\usetikzlibrary{calc,math}
\usepackage{listings}
\usepackage{color}                                            %%
\usepackage{array}                                            %%
\usepackage{longtable}                                        %%
\usepackage{calc}                                             %%
\usepackage{multirow}                                         %%
\usepackage{hhline}                                           %%
\usepackage{ifthen}                                           %%
%optionally (for landscape tables embedded in another document): %%
\usepackage{lscape}     
\usepackage{multicol}
\usepackage{chngcntr}
\usepackage{amsmath}
\usepackage{cleveref}
%\usepackage{enumerate}

%\usepackage{wasysym}
%\newcounter{MYtempeqncnt}
\DeclareMathOperator*{\Res}{Res}
%\renewcommand{\baselinestretch}{2}
\renewcommand\thesection{\arabic{section}}
\renewcommand\thesubsection{\thesection.\arabic{subsection}}
\renewcommand\thesubsubsection{\thesubsection.\arabic{subsubsection}}

\renewcommand\thesectiondis{\arabic{section}}
\renewcommand\thesubsectiondis{\thesectiondis.\arabic{subsection}}
\renewcommand\thesubsubsectiondis{\thesubsectiondis.\arabic{subsubsection}}

% correct bad hyphenation here
\hyphenation{op-tical net-works semi-conduc-tor}
\def\inputGnumericTable{}                                 %%

\lstset{
	%language=C,
	frame=single, 
	breaklines=true,
	columns=fullflexible
}
%\lstset{
%language=tex,
%frame=single, 
%breaklines=true
%}
\usepackage{graphicx}
\usepackage{pgfplots}

\begin{document}
	%
	
	
	\newtheorem{theorem}{Theorem}[section]
	\newtheorem{problem}{Problem}
	\newtheorem{proposition}{Proposition}[section]
	\newtheorem{lemma}{Lemma}[section]
	\newtheorem{corollary}[theorem]{Corollary}
	\newtheorem{example}{Example}[section]
	\newtheorem{definition}[problem]{Definition}
	%\newtheorem{thm}{Theorem}[section] 
	%\newtheorem{defn}[thm]{Definition}
	%\newtheorem{algorithm}{Algorithm}[section]
	%\newtheorem{cor}{Corollary}
	\newcommand{\BEQA}{\begin{eqnarray}}
		\newcommand{\EEQA}{\end{eqnarray}}
	\newcommand{\define}{\stackrel{\triangle}{=}}
	\bibliographystyle{IEEEtran}
	%\bibliographystyle{ieeetr}
	\providecommand{\mbf}{\mathbf}
	\providecommand{\pr}[1]{\ensuremath{\Pr\left(#1\right)}}
	\providecommand{\qfunc}[1]{\ensuremath{Q\left(#1\right)}}
	\providecommand{\sbrak}[1]{\ensuremath{{}\left[#1\right]}}
	\providecommand{\lsbrak}[1]{\ensuremath{{}\left[#1\right.}}
	\providecommand{\rsbrak}[1]{\ensuremath{{}\left.#1\right]}}
	\providecommand{\brak}[1]{\ensuremath{\left(#1\right)}}
	\providecommand{\lbrak}[1]{\ensuremath{\left(#1\right.}}
	\providecommand{\rbrak}[1]{\ensuremath{\left.#1\right)}}
	\providecommand{\cbrak}[1]{\ensuremath{\left\{#1\right\}}}
	\providecommand{\lcbrak}[1]{\ensuremath{\left\{#1\right.}}
	\providecommand{\rcbrak}[1]{\ensuremath{\left.#1\right\}}}
	\theoremstyle{remark}
	\newtheorem{rem}{Remark}
	\newcommand{\sgn}{\mathop{\mathrm{sgn}}}
	\providecommand{\abs}[1]{\left\vert#1\right\vert}
	\providecommand{\res}[1]{\Res\displaylimits_{#1}} 
	\providecommand{\norm}[1]{\left\lVert#1\right\rVert}
	%\providecommand{\norm}[1]{\lVert#1\rVert}
	\providecommand{\mtx}[1]{\mathbf{#1}}
	\providecommand{\mean}[1]{E\left[ #1 \right]}
	\providecommand{\fourier}{\overset{\mathcal{F}}{ \rightleftharpoons}}
	%\providecommand{\hilbert}{\overset{\mathcal{H}}{ \rightleftharpoons}}
	\providecommand{\system}{\overset{\mathcal{H}}{ \longleftrightarrow}}
	%\newcommand{\solution}[2]{\textbf{Solution:}{#1}}
	\newcommand{\solution}{\noindent \textbf{Solution: }}
	\newcommand{\cosec}{\,\text{cosec}\,}
	\providecommand{\dec}[2]{\ensuremath{\overset{#1}{\underset{#2}{\gtrless}}}}
	\newcommand{\myvec}[1]{\ensuremath{\begin{pmatrix}#1\end{pmatrix}}}
	\newcommand{\mydet}[1]{\ensuremath{\begin{vmatrix}#1\end{vmatrix}}}
	%\numberwithin{equation}{section}
	\numberwithin{equation}{subsection}
	%\numberwithin{problem}{section}
	%\numberwithin{definition}{section}
	\makeatletter
	\@addtoreset{figure}{problem}
	\makeatother
	\let\StandardTheFigure\thefigure
	\let\vec\mathbf
	%\renewcommand{\thefigure}{\theproblem.\arabic{figure}}
	\renewcommand{\thefigure}{\theproblem}
	%\setlist[enumerate,1]{before=\renewcommand\theequation{\theenumi.\arabic{equation}}
	%\counterwithin{equation}{enumi}
	%\renewcommand{\theequation}{\arabic{subsection}.\arabic{equation}}
	\def\putbox#1#2#3{\makebox[0in][l]{\makebox[#1][l]{}\raisebox{\baselineskip}[0in][0in]{\raisebox{#2}[0in][0in]{#3}}}}
	\def\rightbox#1{\makebox[0in][r]{#1}}
	\def\centbox#1{\makebox[0in]{#1}}
	\def\topbox#1{\raisebox{-\baselineskip}[0in][0in]{#1}}
	\def\midbox#1{\raisebox{-0.5\baselineskip}[0in][0in]{#1}}
	\vspace{3cm}
	\title{Matrix Theory EE5609\\  Assignment 11}
	\author{Vimal K B\\MTech Artificial Intelligence\\AI20MTECH12001}
	\maketitle
	\newpage
	%\tableofcontents
	\bigskip
	\renewcommand{\thefigure}{\theenumi}
	\renewcommand{\thetable}{\theenumi}
	\begin{abstract}
		This document solves a problem of linear combinations. 
	\end{abstract}
	All the codes for the figure in this document can be found at
	\begin{lstlisting}
		https://github.com/vimalkb007/EE5609/tree/master/Assignment_11
	\end{lstlisting}
	\section{\textbf{Problem}}
	Let $\vec{A}$ be an $m\times n$ matrix with rank $r$. If the linear system $\vec{A}\vec{X} = \vec{b}$ has a solution for each $\vec{b} \in \mathbf{R}^{m}$, then
	\begin{enumerate}
		\item $m=r$
		\item the column space of $\vec{A}$ is a proper subspace of $\mathbf{R}^{m}$ 
		\item the null space of $\vec{A}$ is a non-trivial subspace of $\mathbf{R}^{n}$ whenever $m=n$
		\item $m\geq n$ implies $m=n$
	\end{enumerate}
	
	
	\section{Theorem}
	
	\begin{theorem}\label{thm1}
		Consider the $m\times n$ system Ax = b, with either b $\neq$ 0 or b = 0. We distinguish the following cases:
		\begin{enumerate}
			\item $\textbf{Unique Solution}$: If rank[A,b] = rank(A) = n $\leq$ m, then and only then the system has a unique solution. In this case, indeed as many as $m-n$ equations are redundant. And the solution $\vec{X} = {\vec{A}^{-1}\vec{b}}$. This is called as $\textbf{Exactly Determined}$.
			\item $\textbf{No Solution}$: If rank[A,b] $>$ rank(A) which necessarily implies $\Vec{b} \neq 0$ and m $>$ rank(A), then and only then the system has no solution. This is called as $\textbf{Overdetermined}$.
		\end{enumerate}
	\end{theorem}
	
	
	\section{\textbf{Solution}}
	If the columns of an $m\times n$ matrix $\Vec{A}$ span $\vec{R}^{m}$ then the equation $\vec{A}\vec{x}=\vec{b}$ is consistent for each $\Vec{b}$ in $\vec{R}^{m}$. \\ \\
	The $\textbf{null space}$ of $\vec{A}$ is defined to be 
	
	\begin{align}\label{eq1}
		Null(\vec{A}) = \{ \vec{x} \in \mathbf{R}^{n} \, \vert \, \vec{A}\vec{x} = 0 \}
	\end{align}
	
	
	\\
	
	Let $\vec{A}$ be given as
	
	\begin{align} \label{eq2}
		\vec{A} = \myvec{-3&-2&4 \\ 14&8&-18 \\ 4&2&-4}
	\end{align}
	
	Reduced Row Echelon form is
	
	\begin{align}
		RREF\left(\vec{A}\right) = \myvec{1&0&0 \\ 0&1&0 \\ 0&0&1}
	\end{align}
	$\therefore$ the only possible nullspace of the matrix $\vec{A}$ is $\myvec{0\\0\\0}$.\\
	
	
	Let $\vec{B}$ be given as
	
	\begin{align} \label{eq3}
		\vec{B} = \myvec{-3&-2&4 \\ 14&8&-18 \\ 4&2&-4 \\ 28&16&-36 \\ 8&4&-8}
	\end{align}
	
	Reduced Row Echelon form is
	
	\begin{align}
		RREF\left(\vec{B}\right) = \myvec{1&0&0 \\ 0&1&0 \\ 0&0&1 \\ 0&0&0 \\ 0&0&0}
	\end{align}
	$\therefore$ the rank of matrix $\vec{B} = 3$.
	
	\begin{table}[hp]
		\begin{tabular}{|l|l|}
			\hline
			Options & Observations\\
			\hline
			& \\
			& The rank of any matrix $\vec{A}$ is the dimension of its column space. When  \\
			$m = r$&  the number of rows ($m$) is equal to the rank ($r$) of the matrix, then \\
			&  their linear combination gives us span of $\vec{R}^{m}$.\\
			& \\
			& $\therefore$ This statement is $\textbf{True}$. \\
			& \\
			\hline 
			& \\
			& Any subspace of a vector space $\vec{V}$ other than $\vec{V}$ itself is considered a \\
			the column & proper subspace of $\vec{V}$. Which means that linear combination of $\vec{A}$\\
			space of $\vec{A}$ & will span less than $m$. That will make the resultant\\
			is a proper & $\vec{b}$ span strictly less than $m$.\\ 
			subspace of & But it is given that $\vec{b} \in \mathbf{R}^{m}$, which is contradicting.\\  
			$\mathbf{R}^{m}$ & \\
			& $\therefore$ This statement is $\textbf{False}$. \\
			& \\
			\hline
			& \\
			the null & From $\eqref{eq2}$ we see that even when $m = n$ then also we are getting a\\ 
			space of $\vec{A}$ & trivial nullspace. \\
			is a non-trivial& \\
			subsapce of $\mathbf{R}^{n}$& \\
			whenever $m=n$& $\therefore$ This statement is $\textbf{False}$. \\
			& \\
			\hline
			& \\
			& It is given that the number of rows are greater than the column, and it \\
			$m \geq n$ & is given that there exists a solution. If we refer to theorem $\eqref{thm1}$ we \\
			implies & see that the corresponding system will be $\textbf{Exactly Determined}$ system. \\
			$m=n$ & \\
			& As an example, it will look like $\eqref{eq3}$.\\
			& \\
			& $\therefore$ This statement is $\textbf{True}$. \\
			& \\
			\hline
		\end{tabular}
	\end{table}
	
\end{document}