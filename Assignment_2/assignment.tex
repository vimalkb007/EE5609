\documentclass[journal,12pt,twocolumn]{IEEEtran}

\usepackage{setspace}
\usepackage{gensymb}

\singlespacing


\usepackage[cmex10]{amsmath}

\usepackage{amsthm}

\usepackage{mathrsfs}
\usepackage{txfonts}
\usepackage{stfloats}
\usepackage{bm}
\usepackage{cite}
\usepackage{cases}
\usepackage{subfig}

\usepackage{longtable}
\usepackage{multirow}

\usepackage{enumitem}
\usepackage{mathtools}
\usepackage{steinmetz}
\usepackage{tikz}
\usepackage{circuitikz}
\usepackage{verbatim}
\usepackage{tfrupee}
\usepackage[breaklinks=true]{hyperref}
\usepackage{graphicx}
\usepackage{tkz-euclide}

\usetikzlibrary{calc,math}
\usepackage{listings}
\usepackage{color}                                            %%
\usepackage{array}                                            %%
\usepackage{longtable}                                        %%
\usepackage{calc}                                             %%
\usepackage{multirow}                                         %%
\usepackage{hhline}                                           %%
\usepackage{ifthen}                                           %%
\usepackage{lscape}     
\usepackage{multicol}
\usepackage{chngcntr}

\DeclareMathOperator*{\Res}{Res}

\renewcommand\thesection{\arabic{section}}
\renewcommand\thesubsection{\thesection.\arabic{subsection}}
\renewcommand\thesubsubsection{\thesubsection.\arabic{subsubsection}}

\renewcommand\thesectiondis{\arabic{section}}
\renewcommand\thesubsectiondis{\thesectiondis.\arabic{subsection}}
\renewcommand\thesubsubsectiondis{\thesubsectiondis.\arabic{subsubsection}}


\hyphenation{op-tical net-works semi-conduc-tor}
\def\inputGnumericTable{}                                 %%

\lstset{
	%language=C,
	frame=single, 
	breaklines=true,
	columns=fullflexible
}
\begin{document}
	
	
	\newtheorem{theorem}{Theorem}[section]
	\newtheorem{problem}{Problem}
	\newtheorem{proposition}{Proposition}[section]
	\newtheorem{lemma}{Lemma}[section]
	\newtheorem{corollary}[theorem]{Corollary}
	\newtheorem{example}{Example}[section]
	\newtheorem{definition}[problem]{Definition}
	
	\newcommand{\BEQA}{\begin{eqnarray}}
		\newcommand{\EEQA}{\end{eqnarray}}
	\newcommand{\define}{\stackrel{\triangle}{=}}
	\bibliographystyle{IEEEtran}
	\providecommand{\mbf}{\mathbf}
	\providecommand{\pr}[1]{\ensuremath{\Pr\left(#1\right)}}
	\providecommand{\qfunc}[1]{\ensuremath{Q\left(#1\right)}}
	\providecommand{\sbrak}[1]{\ensuremath{{}\left[#1\right]}}
	\providecommand{\lsbrak}[1]{\ensuremath{{}\left[#1\right.}}
	\providecommand{\rsbrak}[1]{\ensuremath{{}\left.#1\right]}}
	\providecommand{\brak}[1]{\ensuremath{\left(#1\right)}}
	\providecommand{\lbrak}[1]{\ensuremath{\left(#1\right.}}
	\providecommand{\rbrak}[1]{\ensuremath{\left.#1\right)}}
	\providecommand{\cbrak}[1]{\ensuremath{\left\{#1\right\}}}
	\providecommand{\lcbrak}[1]{\ensuremath{\left\{#1\right.}}
	\providecommand{\rcbrak}[1]{\ensuremath{\left.#1\right\}}}
	\theoremstyle{remark}
	\newtheorem{rem}{Remark}
	\newcommand{\sgn}{\mathop{\mathrm{sgn}}}
	\providecommand{\abs}[1]{\left\vert#1\right\vert}
	\providecommand{\res}[1]{\Res\displaylimits_{#1}} 
	\providecommand{\norm}[1]{\left\lVert#1\right\rVert}
	%\providecommand{\norm}[1]{\lVert#1\rVert}
	\providecommand{\mtx}[1]{\mathbf{#1}}
	\providecommand{\mean}[1]{E\left[ #1 \right]}
	\providecommand{\fourier}{\overset{\mathcal{F}}{ \rightleftharpoons}}
	%\providecommand{\hilbert}{\overset{\mathcal{H}}{ \rightleftharpoons}}
	\providecommand{\system}{\overset{\mathcal{H}}{ \longleftrightarrow}}
	%\newcommand{\solution}[2]{\textbf{Solution:}{#1}}
	\newcommand{\solution}{\noindent \textbf{Solution: }}
	\newcommand{\cosec}{\,\text{cosec}\,}
	\providecommand{\dec}[2]{\ensuremath{\overset{#1}{\underset{#2}{\gtrless}}}}
	\newcommand{\myvec}[1]{\ensuremath{\begin{pmatrix}#1\end{pmatrix}}}
	\newcommand{\mydet}[1]{\ensuremath{\begin{vmatrix}#1\end{vmatrix}}}
	\numberwithin{equation}{subsection}
	\makeatletter
	\@addtoreset{figure}{problem}
	\makeatother
	\let\StandardTheFigure\thefigure
	\let\vec\mathbf
	\renewcommand{\thefigure}{\theproblem}
	\def\putbox#1#2#3{\makebox[0in][l]{\makebox[#1][l]{}\raisebox{\baselineskip}[0in][0in]{\raisebox{#2}[0in][0in]{#3}}}}
	\def\rightbox#1{\makebox[0in][r]{#1}}
	\def\centbox#1{\makebox[0in]{#1}}
	\def\topbox#1{\raisebox{-\baselineskip}[0in][0in]{#1}}
	\def\midbox#1{\raisebox{-0.5\baselineskip}[0in][0in]{#1}}
	\vspace{3cm}



\title{EE5609 Assignment 2}
\author{Vimal K B \\Roll No - AI20MTECH14002}

\maketitle
\newpage
%\tableofcontents
\bigskip

\renewcommand{\thefigure}{\theenumi}
\renewcommand{\thetable}{\theenumi}

\begin{abstract}
	This assignment involves finding the angle between a given pair of lines 
\end{abstract}

The python solution code for this problem can be downloaded from

\begin{lstlisting}
	https://github.com/vimalkb007/EE5609/blob/master/Assignment_2/codes/assignment2_solution.py
\end{lstlisting}

The python verification code for this problem can be downloaded from

\begin{lstlisting}
	https://github.com/vimalkb007/EE5609/blob/master/Assignment_2/codes/assignment2_solution_verify.py
\end{lstlisting}

\section{\textbf{Problem Statement}}
	Find the angle between the following pair of lines
	\begin{enumerate}
		\item 
		\begin{align}\label{eq1}
			\frac{x-2}{2} = \frac{y-1}{5} &= \frac{z+3}{-3}, 
			\\
			\frac{x+2}{-1} = \frac{y-4}{8} &= \frac{z-5}{4} 
		\end{align}
		\item 
		\begin{align}\label{eq2}
			\frac{x}{2} = \frac{y}{2} &= \frac{z}{1}, 
			\\
			\frac{x-5}{4} = \frac{y-2}{1} &= \frac{z-3}{8} 
		\end{align}
	\end{enumerate}
	
\section{\textbf{Theory}}

	Given two symmetric line equations we can represent them in the vector format.
	
	Using the dot product of the vectors we can find the angle between the two lines. If we have two vectors $\vec{u}$, $\vec{v}$, the dot product is given by 
	
	\begin{equation}\label{eq3}
	%	\vec{u}^{T}\vec{v} = \norm{$\vec{u}$}\norm{$\vec{v}$} \cos \theta	
	\vec{u}^{T} \vec{v} = \norm{\vec{u}} \norm{\vec{v}} \cos \theta
	\end{equation}
	
	From the above \ref{eq3} we get the angle between two vectors as
	
	\begin{equation}\label{eq4}
		\theta = \cos ^{-1}\frac{\vec{u}^{T}\vec{v}}{\norm{\vec{u}} \norm{\vec{v}}}
	\end{equation}
	
	
\section{\textbf{Solution}}
From theory, we understand that using dot product we can find the angle between the lines 
\begin{enumerate}
	\item 
	\begin{align}\label{eq5}
		\frac{x-2}{2} = \frac{y-1}{5} &= \frac{z+3}{-3}, 
	\end{align}
	\begin{align}\label{eq6}
		\frac{x+2}{-1} = \frac{y-4}{8} &= \frac{z-5}{4} 
	\end{align}


The above symmetric equations \ref{eq5}, \ref{eq6} can be represented in the vector form as 
\begin{align}\label{eq7}
	\quad \vec{r_1} &= \myvec{2\\1\\-3} + \lambda_1\myvec{2\\5\\-3}
	\\
	\quad \vec{r_2} &= \myvec{-2\\4\\5} + \lambda_2\myvec{-1\\8\\4}
\end{align}

As we have to find the angle between the vectors, we will only be taking the direction vectors into consideration. The direction vectors are $\vec{u}$ = $\myvec{2\\5\\-3}$ and $\vec{v}$ = $\myvec{-1\\8\\4}$. We can find the corresponding magnitude values

\begin{align}\label{eq9}
	\norm{\vec{u}} =\sqrt{2^{2}+5^{2}+(-3)^{2}} =\sqrt{38}
\end{align}
\begin{align}\label{eq10}
	\norm{\vec{v}} =\sqrt{(-1)^{2}+8^{2}+4^{2}} =\sqrt{81}
\end{align}

Using \ref{eq4}, \ref{eq9}, \ref{eq10} we get
\begin{align}
	\theta = \cos ^{-1}\frac{\myvec{2\\5\\-3}^{T}\myvec{-1\\8\\4}}{(\sqrt{38})(\sqrt{81})} 
	\\
	\theta = \cos ^{-1}\frac{26}{55.4797}
	\\
	\theta = \cos ^{-1} (0.4686)
	\\
	\theta = 62.053\degree
\end{align}

Therefore, the angle between the two lines is $62.053\degree$.See Fig. \ref{fig:line_equation_1}

\begin{figure}
	\centering
	\includegraphics[width=\columnwidth]{./codes/figs/Line_interest_1.png}
	\caption{Graph for equations \ref{eq7}}
	\label{fig:line_equation_1}
\end{figure}


	\item 
	\begin{align}\label{eq12}
		\frac{x}{2} = \frac{y}{2} &= \frac{z}{1}, 
	\end{align}
	\begin{align}\label{eq13}
		\frac{x-5}{4} = \frac{y-2}{1} &= \frac{z-3}{8} 
	\end{align}



The above symmetric equations \ref{eq12}, \ref{eq13} can be represented in the vector form as 
\begin{align}\label{eq14}
	\quad \vec{r_1} &= \myvec{0\\0\\0} + \lambda_1\myvec{2\\2\\1}
	\\
	\quad \vec{r_2} &= \myvec{5\\2\\3} + \lambda_2\myvec{4\\1\\8}
\end{align}

As we have to find the angle between the vectors, we will only be taking the direction vectors into consideration. The direction vectors are $\vec{u}$ = $\myvec{2\\2\\1}$ and $\vec{v}$ = $\myvec{4\\1\\8}$. We can find the corresponding magnitude values

\begin{align}\label{eq16}
	\norm{\vec{u}} =\sqrt{2^{2}+2^{2}+1^{2}} =\sqrt{9}
\end{align}
\begin{align}\label{eq17}
	\norm{\vec{v}} =\sqrt{4^{2}+1^{2}+8^{2}} =\sqrt{81}
\end{align}

Using \ref{eq4}, \ref{eq16}, \ref{eq17} we get
\begin{align}
	\theta = \cos ^{-1}\frac{\myvec{2\\2\\1}^{T}\myvec{4\\1\\8}}{(\sqrt{9})(\sqrt{81})} 
	\\
	\theta = \cos ^{-1}\frac{18}{27.00}
	\\
	\theta = \cos ^{-1} (0.667)
	\\
	\theta = 48.189\degree
\end{align}

Therefore, the angle between the two lines is $48.189\degree$. See Fig. \ref{fig:line_equation_2}


\begin{figure}
	\centering
	\includegraphics[width=\columnwidth]{./codes/figs/Line_interest_2.png}
	\caption{Graph for equations \ref{eq14}}
	\label{fig:line_equation_2}
\end{figure}
\end{enumerate}

    
\end{document}