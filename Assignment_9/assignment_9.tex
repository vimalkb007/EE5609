\documentclass[journal,12pt,twocolumn]{IEEEtran}

\usepackage{setspace}
\usepackage{gensymb}

\singlespacing


\usepackage[cmex10]{amsmath}

\usepackage{amsthm}

\usepackage{mathrsfs}
\usepackage{txfonts}
\usepackage{stfloats}
\usepackage{bm}
\usepackage{cite}
\usepackage{cases}
\usepackage{subfig}

\usepackage{longtable}
\usepackage{multirow}

\usepackage{enumitem}
\usepackage{mathtools}
\usepackage{steinmetz}
\usepackage{tikz}
\usepackage{circuitikz}
\usepackage{verbatim}
\usepackage{tfrupee}
\usepackage[breaklinks=true]{hyperref}

\usepackage{tkz-euclide}

\usetikzlibrary{calc,math}
\usepackage{listings}
\usepackage{color}                                            %%
\usepackage{array}                                            %%
\usepackage{longtable}                                        %%
\usepackage{calc}                                             %%
\usepackage{multirow}                                         %%
\usepackage{hhline}                                           %%
\usepackage{ifthen}                                           %%
\usepackage{lscape}     
\usepackage{multicol}
\usepackage{chngcntr}

\DeclareMathOperator*{\Res}{Res}

\renewcommand\thesection{\arabic{section}}
\renewcommand\thesubsection{\thesection.\arabic{subsection}}
\renewcommand\thesubsubsection{\thesubsection.\arabic{subsubsection}}

\renewcommand\thesectiondis{\arabic{section}}
\renewcommand\thesubsectiondis{\thesectiondis.\arabic{subsection}}
\renewcommand\thesubsubsectiondis{\thesubsectiondis.\arabic{subsubsection}}


\hyphenation{op-tical net-works semi-conduc-tor}
\def\inputGnumericTable{}                                 %%

\lstset{
	%language=C,
	frame=single, 
	breaklines=true,
	columns=fullflexible
}
\begin{document}
	
	
	\newtheorem{theorem}{Theorem}[section]
	\newtheorem{problem}{Problem}
	\newtheorem{proposition}{Proposition}[section]
	\newtheorem{lemma}{Lemma}[section]
	\newtheorem{corollary}[theorem]{Corollary}
	\newtheorem{example}{Example}[section]
	\newtheorem{definition}[problem]{Definition}
	
	\newcommand{\BEQA}{\begin{eqnarray}}
		\newcommand{\EEQA}{\end{eqnarray}}
	\newcommand{\define}{\stackrel{\triangle}{=}}
	\bibliographystyle{IEEEtran}
	\providecommand{\mbf}{\mathbf}
	\providecommand{\pr}[1]{\ensuremath{\Pr\left(#1\right)}}
	\providecommand{\qfunc}[1]{\ensuremath{Q\left(#1\right)}}
	\providecommand{\sbrak}[1]{\ensuremath{{}\left[#1\right]}}
	\providecommand{\lsbrak}[1]{\ensuremath{{}\left[#1\right.}}
	\providecommand{\rsbrak}[1]{\ensuremath{{}\left.#1\right]}}
	\providecommand{\brak}[1]{\ensuremath{\left(#1\right)}}
	\providecommand{\lbrak}[1]{\ensuremath{\left(#1\right.}}
	\providecommand{\rbrak}[1]{\ensuremath{\left.#1\right)}}
	\providecommand{\cbrak}[1]{\ensuremath{\left\{#1\right\}}}
	\providecommand{\lcbrak}[1]{\ensuremath{\left\{#1\right.}}
	\providecommand{\rcbrak}[1]{\ensuremath{\left.#1\right\}}}
	\theoremstyle{remark}
	\newtheorem{rem}{Remark}
	\newcommand{\sgn}{\mathop{\mathrm{sgn}}}
	\providecommand{\abs}[1]{\left\vert#1\right\vert}
	\providecommand{\res}[1]{\Res\displaylimits_{#1}} 
	\providecommand{\norm}[1]{\left\lVert#1\right\rVert}
	%\providecommand{\norm}[1]{\lVert#1\rVert}
	\providecommand{\mtx}[1]{\mathbf{#1}}
	\providecommand{\mean}[1]{E\left[ #1 \right]}
	\providecommand{\fourier}{\overset{\mathcal{F}}{ \rightleftharpoons}}
	%\providecommand{\hilbert}{\overset{\mathcal{H}}{ \rightleftharpoons}}
	\providecommand{\system}{\overset{\mathcal{H}}{ \longleftrightarrow}}
	%\newcommand{\solution}[2]{\textbf{Solution:}{#1}}
	\newcommand{\solution}{\noindent \textbf{Solution: }}
	\newcommand{\cosec}{\,\text{cosec}\,}
	\providecommand{\dec}[2]{\ensuremath{\overset{#1}{\underset{#2}{\gtrless}}}}
	\newcommand{\myvec}[1]{\ensuremath{\begin{pmatrix}#1\end{pmatrix}}}
	\newcommand{\mydet}[1]{\ensuremath{\begin{vmatrix}#1\end{vmatrix}}}
	\numberwithin{equation}{subsection}
	\makeatletter
	\@addtoreset{figure}{problem}
	\makeatother
	\let\StandardTheFigure\thefigure
	\let\vec\mathbf
	\renewcommand{\thefigure}{\theproblem}
	\def\putbox#1#2#3{\makebox[0in][l]{\makebox[#1][l]{}\raisebox{\baselineskip}[0in][0in]{\raisebox{#2}[0in][0in]{#3}}}}
	\def\rightbox#1{\makebox[0in][r]{#1}}
	\def\centbox#1{\makebox[0in]{#1}}
	\def\topbox#1{\raisebox{-\baselineskip}[0in][0in]{#1}}
	\def\midbox#1{\raisebox{-0.5\baselineskip}[0in][0in]{#1}}
	\vspace{3cm}
	\title{EE5609: Matrix Theory\\
		Assignment 9\\}
	\author{Vimal K B\\
		AI20MTECH12001}
	\maketitle
	\newpage
	\bigskip
	\renewcommand{\thefigure}{\theenumi}
	\renewcommand{\thetable}{\theenumi}
	\begin{abstract}
		This document explains the concept of vector space over complex numbers.
	\end{abstract}
	Download all solutions from 
	\begin{lstlisting}
		https://github.com/vimalkb007/EE5609/tree/master/Assignment_9
	\end{lstlisting}
	%
	%
	\section{Problem}
	Let $\vec{V}$ be the vector space over the complex numbers of all functions from $\mathbf{R}$ into $\mathbf{C}$, i.e. the space of all complex-valued functions on the real line. Let $f_1(x) = 1$, $f_2(x) = e^{ix}$, $f_3(x) = e^{-ix}$.\\
	
	\begin{enumerate}[label=(\alph*)]
		\item Prove that $f_1$, $f_2$, and $f_3$ are linearly dependent.\\
		\item Let $g_1(x) = 1$, $g_2(x) = \cos(x)$, $g_3(x) = \sin(x)$. Find an invertible $3\times 3$ matrix $P$ such that\\
		$g_i = \sum_{i=1}^{3} P_{ii}f_i$
	\end{enumerate}
	
	\section{Solution}
	
	\begin{enumerate}[label=(\alph*)]
		\item To check for independence, lets represent the function in a polynomial format as \\
		
		\begin{align}
			\alpha f_1 + \beta f_2 + \gamma f_3 = 0\\
			\alpha + \beta e^{ix} + \gamma e^{-ix} = 0 \label{eq1}
		\end{align}
		
		Multiply the whole equation with $e^{ix}$ to get $\beta (e^{ix})^{2} + \alpha e^{ix} + \gamma = 0$.\\ 
		
		Let $y = e^{ix}$, which makes the equation as $\beta y^{2} + \alpha y + \gamma = 0$. The above quadratic polynomial in $y$ can be zero for atmost two values of $y$. But $y = e^{ix}$, and $e^{ix}$ takes infinitely many different values as x varies in $\mathbf{R}$. So $\eqref{eq1}$ cannot be zero for all $y = e^{ix}$. Which implies there is a contradiction. \\
		
		Then the only case of $\alpha = \beta = \gamma = 0$, can satisfy $\eqref{eq1}$. Therefore, $f_1, f_2, f_3$ are linearly independent. \\
		
		\item We need to find the coordinates of vectors $g_i$ where $i = 1, 2, 3$ in ordered basis
		\begin{align}
			B = \myvec{f_1 & f_2 & f_3} \label{eq5}
		\end{align}
		
		It is given that $g_1 = 1$, which can be written as 
		\begin{align}
			g_1 = f_1 \label{eq4}   \\ 
			\implies g_1 = \myvec{f_1 & f_2 & f_3}\myvec{1\\0\\0}
		\end{align}
		
		We can use the following identities:-
		
		\begin{align}
			\cos (x) = \frac{1}{2}e^{ix} + \frac{1}{2}e^{-ix} \label{eq2}\\
			\sin (x) = \frac{1}{2i}e^{ix} - \frac{1}{2i}e^{-ix} \label{eq3}
		\end{align}
		
		Comparing equations \eqref{eq2} and \eqref{eq3} with $f_2, f_3$, we can write $g_2$ and $g_3$ as 
		
		\begin{align}
			g_2 = \frac{1}{2}f_2 + \frac{1}{2}f_3 \\
			\implies g_2 = \myvec{f_1 & f_2 & f_3} \myvec{0\\\frac{1}{2}\\\frac{1}{2}} \\
			g_3 = \frac{1}{2i}f_2 - \frac{1}{2i}f_3 \\
			\implies g_3 = \myvec{f_1 & f_2 & f_3} \myvec{0\\\frac{1}{2i}\\\frac{-1}{2i}}
		\end{align}
		
		Therefore, the required matrix $P$ is
		\begin{align}
			P = \myvec{1&0&0\\0&\frac{1}{2}&\frac{1}{2i}\\0&\frac{1}{2}&\frac{-1}{2i}}
		\end{align}
		
	\end{enumerate}
	
\end{document}