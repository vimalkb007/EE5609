\documentclass[journal,12pt,twocolumn]{IEEEtran}
%
\usepackage{xparse,mathtools}
\usepackage{setspace}
\usepackage{gensymb}
%\doublespacing
\singlespacing

\usepackage{esvect}
\usepackage{graphicx}
%\usepackage{amssymb}
%\usepackage{relsize}
%\usepackage[cmex10]{amsmath}
%\usepackage{amsthm}
%\interdisplaylinepenalty=2500
%\savesymbol{iint}
%\usepackage{txfonts}
%\restoresymbol{TXF}{iint}
%\usepackage{wasysym}
\usepackage{amsthm}
%\usepackage{iithtlc}
\usepackage{mathrsfs}
\usepackage{txfonts}
\usepackage{stfloats}
\usepackage{bm}
\usepackage{cite}
\usepackage{cases}
\usepackage{subfig}
%\usepackage{xtab}
\usepackage{longtable}
\usepackage{multirow}
%\usepackage{algorithm}
%\usepackage{algpseudocode}
\usepackage{enumitem}
\usepackage{mathtools}
\usepackage{steinmetz}
\usepackage{tikz}
\usepackage{circuitikz}
\usepackage{verbatim}
\usepackage{tfrupee}
\usepackage[breaklinks=true]{hyperref}
%\usepackage{stmaryrd}
\usepackage{tkz-euclide} % loads  TikZ and tkz-base
%\usetkzobj{all}
\usetikzlibrary{calc,math}
\usepackage{listings}
\usepackage{color}                                            %%
\usepackage{array}                                            %%
\usepackage{longtable}                                        %%
\usepackage{calc}                                             %%
\usepackage{multirow}                                         %%
\usepackage{hhline}                                           %%
\usepackage{ifthen}                                           %%
%optionally (for landscape tables embedded in another document): %%
\usepackage{lscape}     
\usepackage{multicol}
\usepackage{chngcntr}
%\usepackage{enumerate}

%\usepackage{wasysym}
%\newcounter{MYtempeqncnt}
\DeclareMathOperator*{\Res}{Res}
%\renewcommand{\baselinestretch}{2}
\renewcommand\thesection{\arabic{section}}
\renewcommand\thesubsection{\thesection.\arabic{subsection}}
\renewcommand\thesubsubsection{\thesubsection.\arabic{subsubsection}}

\renewcommand\thesectiondis{\arabic{section}}
\renewcommand\thesubsectiondis{\thesectiondis.\arabic{subsection}}
\renewcommand\thesubsubsectiondis{\thesubsectiondis.\arabic{subsubsection}}

% correct bad hyphenation here
\hyphenation{op-tical net-works semi-conduc-tor}
\def\inputGnumericTable{}                                 %%

\lstset{
	%language=C,
	frame=single, 
	breaklines=true,
	columns=fullflexible
}
%\lstset{
%language=tex,
%frame=single, 
%breaklines=true
%}



\begin{document}
\newtheorem{theorem}{Theorem}[section]
\newtheorem{problem}{Problem}
\newtheorem{proposition}{Proposition}[section]
\newtheorem{lemma}{Lemma}[section]
\newtheorem{corollary}[theorem]{Corollary}
\newtheorem{example}{Example}[section]
\newtheorem{definition}[problem]{Definition}
%\newtheorem{thm}{Theorem}[section] 
%\newtheorem{defn}[thm]{Definition}
%\newtheorem{algorithm}{Algorithm}[section]
%\newtheorem{cor}{Corollary}
\newcommand{\BEQA}{\begin{eqnarray}}
	\newcommand{\EEQA}{\end{eqnarray}}
\newcommand{\define}{\stackrel{\triangle}{=}}
\bibliographystyle{IEEEtran}
%\bibliographystyle{ieeetr}
\providecommand{\mbf}{\mathbf}
\providecommand{\pr}[1]{\ensuremath{\Pr\left(#1\right)}}
\providecommand{\qfunc}[1]{\ensuremath{Q\left(#1\right)}}
\providecommand{\sbrak}[1]{\ensuremath{{}\left[#1\right]}}
\providecommand{\lsbrak}[1]{\ensuremath{{}\left[#1\right.}}
\providecommand{\rsbrak}[1]{\ensuremath{{}\left.#1\right]}}
\providecommand{\brak}[1]{\ensuremath{\left(#1\right)}}
\providecommand{\lbrak}[1]{\ensuremath{\left(#1\right.}}
\providecommand{\rbrak}[1]{\ensuremath{\left.#1\right)}}
\providecommand{\cbrak}[1]{\ensuremath{\left\{#1\right\}}}
\providecommand{\lcbrak}[1]{\ensuremath{\left\{#1\right.}}
\providecommand{\rcbrak}[1]{\ensuremath{\left.#1\right\}}}
\theoremstyle{remark}
\newtheorem{rem}{Remark}
\newcommand{\sgn}{\mathop{\mathrm{sgn}}}
\providecommand{\abs}[1]{\left\vert#1\right\vert}
\providecommand{\res}[1]{\Res\displaylimits_{#1}} 
\providecommand{\norm}[1]{\left\lVert#1\right\rVert}
%\providecommand{\norm}[1]{\lVert#1\rVert}
\providecommand{\mtx}[1]{\mathbf{#1}}
\providecommand{\mean}[1]{E\left[ #1 \right]}
\providecommand{\fourier}{\overset{\mathcal{F}}{ \rightleftharpoons}}
%\providecommand{\hilbert}{\overset{\mathcal{H}}{ \rightleftharpoons}}
\providecommand{\system}{\overset{\mathcal{H}}{ \longleftrightarrow}}
%\newcommand{\solution}[2]{\textbf{Solution:}{#1}}
\newcommand{\solution}{\noindent \textbf{Solution: }}
\newcommand{\cosec}{\,\text{cosec}\,}
\providecommand{\dec}[2]{\ensuremath{\overset{#1}{\underset{#2}{\gtrless}}}}
\newcommand{\myvec}[1]{\ensuremath{\begin{pmatrix}#1\end{pmatrix}}}
\newcommand{\mydet}[1]{\ensuremath{\begin{vmatrix}#1\end{vmatrix}}}
%\numberwithin{equation}{section}
\numberwithin{equation}{subsection}
%\numberwithin{problem}{section}
%\numberwithin{definition}{section}
\makeatletter
\@addtoreset{figure}{problem}
\makeatother
\let\StandardTheFigure\thefigure
\let\vec\mathbf
%\renewcommand{\thefigure}{\theproblem.\arabic{figure}}
\renewcommand{\thefigure}{\theproblem}
%\setlist[enumerate,1]{before=\renewcommand\theequation{\theenumi.\arabic{equation}}
%\counterwithin{equation}{enumi}
%\renewcommand{\theequation}{\arabic{subsection}.\arabic{equation}}
\def\putbox#1#2#3{\makebox[0in][l]{\makebox[#1][l]{}\raisebox{\baselineskip}[0in][0in]{\raisebox{#2}[0in][0in]{#3}}}}
\def\rightbox#1{\makebox[0in][r]{#1}}
\def\centbox#1{\makebox[0in]{#1}}
\def\topbox#1{\raisebox{-\baselineskip}[0in][0in]{#1}}
\def\midbox#1{\raisebox{-0.5\baselineskip}[0in][0in]{#1}}
\vspace{3cm}


\title{EE5609 Assignment 1}
\author{Vimal K B \\Roll No - AI20MTECH14002}
\date{\today}

\maketitle
\newpage
%\tableofcontents
\bigskip

\renewcommand{\thefigure}{\theenumi}
\renewcommand{\thetable}{\theenumi}

\begin{abstract}
	This assignment involves finding a vector which is perpendicular to given two vectors and non-perpendicular to a third vector. 
\end{abstract}

The python solution code for this problem can be downloaded from

\begin{lstlisting}
	https://github.com/vimalkb007/EE5609/blob/master/Assignment_1/codes/assignment1_solution.py
\end{lstlisting}

The python verification code for this problem can be downloaded from

\begin{lstlisting}
	https://github.com/vimalkb007/EE5609/blob/master/Assignment_1/codes/assignment1_solution_verify.py
\end{lstlisting}

\section{\textbf{Problem Statement}}
	Let 
	$
	\vec{a}=\myvec{1\\4\\2},
	\vec{b}=\myvec{3\\-2\\7} \text{ and }
	\vec{c}=\myvec{2\\-1\\4}.
	$
	Find a vector $\vec{d}$ such that $\vec{d}\perp\vec{a},\vec{d}\perp\vec{b}$ and $\vec{d}^T\vec{c} = 15$.
	
\section{\textbf{Theory}}

	If two vectors are perpendicular, then their dot product is 0.
	If we have two vectors $\vec{x}$, $\vec{y}$ is given by 
		$\vec{x}$$\cdot$$\vec{y}$ = $|$$\vec{x}$$|$$|$$\vec{y}$$|$$\cos(\theta)$.
	
	When $\theta$ = $\pi/2$ $\;$$(90^\circ)$,then $\;$ $\cos$ $\theta$  = 0 $\implies$ $\vec{x}$$\cdot$$\vec{y}$ = 0.
	
	If we have 3 equations and 3 unknowns, we can use Guassian Elimination method in order to find the unknowns.
	
\section{\textbf{Solution}}

	Lets consider vector $\vec{d}$ as $\myvec{d_1\\d_2\\d_3}$.\hfill \break
	It is given that $\vec{d}\perp\vec{a}$, then their correponding dot product will be 0.\hfill \break
		$\vec{d}$$^{T}$$\vec{a}$ = 0 $\implies$ $\myvec{d_1\\d_2\\d_3}^ T$ $\myvec{1\\4\\2}$ = 0 \hfill \break
		

	\begin{equation}\label{eq1}
		d_1 + 4d_2 + 2d_3 = 0
	\end{equation}

	Similarly, as $\vec{d}\perp\vec{b}$, \hfill \break
	$\vec{d}$$^{T}$$\vec{b}$ = 0 $\implies$ $\myvec{d_1\\d_2\\d_3}^T$ $\myvec{3\\-2\\7}$ = 0 
	\begin{equation}\label{eq2}
		3d_1 - 2d_2 + 7d_3 = 0
	\end{equation} 

	Since, it is given that $\vec{d}^T\vec{c} = 15$, we can write it as 
	($\begin{matrix}d_1 & d_2 & d_3\end{matrix}$)  $\myvec{2\\-1\\4}$ = 15.
	
	\begin{equation}\label{eq3}
		2d_1 - d_2 + 4d_3 = 15
	\end{equation} 
    
    
    
    Using equations \ref{eq1}, \ref{eq2}, \ref{eq3}, we can write them in a Matrix Representation of Linear Equations $A$$x$=$B$ form as:
    
    \[
    \begin{bmatrix}
    	1 & 4 & 2 \\
    	3 & -2 & 7 \\
    	2 & -1 & 4 
    \end{bmatrix}
    \begin{bmatrix}
    	d_1 \\ d_2 \\ d_3 
    \end{bmatrix}
    =
    \begin{bmatrix}
    	0 \\ 0 \\ 15
    \end{bmatrix}
    \]
     
    
    we can use Guassian Elimination Method in order to find the values of $d_1$, $d_2$, $d_3$.
    
    

	\begin{align}
		\myvec{
			1 & 4 & 2 & \vrule & 0 \\
			3 & -2 & 7 & \vrule & 0 \\
			2 & -1 & 4 & \vrule & 15 \\
		}
		\\
		\xleftrightarrow[R_2 \leftarrow R_2-3R_1]{R_3 \leftarrow R_3 - 2R_1}
		\myvec{
			1 & 4 & 2 & \vrule & 0 \\
			0 & -14 & 1 & \vrule & 0 \\
			0 & -9 & 0 & \vrule & 15
		}
		\\
		\xleftrightarrow[]{R_3\leftarrow R_3-\frac{9}{14}R_2}
		\myvec{
			1 & 4 & 2 & \vrule & 0 \\
			0 & -14 & 1 & \vrule & 0 \\
			0 & 0 & \frac{-9}{14} & \vrule & 15
		}\\
		\xleftrightarrow[R_2 \leftarrow \frac{-1}{14}R_2]{R_3 \leftarrow \frac{-14}{9}R_2}
		\myvec{
			1 & 4 & 2 & \vrule & 0 \\[0.2cm]
			0 & 1 & \frac{-1}{14} & \vrule & 0 \\[0.2cm]
			0 & 0 & 1 & \vrule & \frac{-210}{9}
		}\\
		\xleftrightarrow[]{R_1 \leftarrow R_1+\frac{1}{14}R_3}
		\myvec{
			1 & 4 & 2 & \vrule & 0 \\[0.2cm]
			0 & 1 & 0 & \vrule & \frac{-210}{126} \\[0.2cm]
			0 & 0 & 1 & \vrule & \frac{-210}{9}
		}
	\end{align}


	\begin{align}
		\xleftrightarrow[]{R_1 \leftarrow R_1-4R_3}
		\myvec{
			1 & 0 & 2 & \vrule & \frac{840}{126} \\[0.2cm]
			0 & 1 & 0 & \vrule & \frac{-210}{126} \\[0.2cm]
			0 & 0 & 1 & \vrule & \frac{-210}{9}
		}\\
		\xleftrightarrow[]{R_1 \leftarrow R_1-2R_3}
		\myvec{
			1 & 0 & 0 & \vrule & \frac{6720}{126} \\[0.2cm]
			0 & 1 & 0 & \vrule & \frac{-210}{126} \\[0.2cm]
			0 & 0 & 1 & \vrule & \frac{-210}{9}
		}
\end{align}


By using Guassian Elimination Method, we were able to get the vector $\vec{d}$ as
	$\myvec{\frac{6720}{126} \\[0.2cm]\frac{-210}{126} \\[0.2cm]\frac{-210}{9}}$\\[0.5cm]
	
The resultant vector $\vec{d}$ = $\myvec{53.333\\-1.667\\-23.333}$
    
\end{document}